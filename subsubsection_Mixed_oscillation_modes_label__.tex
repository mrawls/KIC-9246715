\subsubsection{Mixed oscillation modes}
\label{subsubsec_mixed}
Based on the distribution of mixed $\ell = 1$ modes, \citet{gau14} reported that the oscillation pattern period spacing was typical of that of a star from the secondary red clump, i.e., a core-He-burning star that has not experienced a helium flash. However, this conclusion was based on a \revise{dipole gravity mode} period spacing of $\Delta \Pi_1 \simeq 150 \ \rm{sec}$, which is difficult to measure robustly from a noisy oscillation spectrum. Red giant branch stars have smaller period spacings than red clump stars, and ($\Delta \Pi_1 = 150 \ \rm{sec}$, $\Delta \nu = 8.31 \ \mu \rm{Hz}$) puts the oscillating star on the very edge of the asteroseismic parameter space that defines the secondary red clump \citep{mos14}.
%Therefore, while asteroseismology does indicate the oscillator in KIC 9246715 is a red clump star, there is a large uncertainty attached to the classification.

\revise{To more accurately assess the evolutionary stage of KIC 9246715, we perform a Bayesian fit to the individual oscillation modes of the star using the \textsc{D\large{iamonds}} code \citep{cor14} and the methodology for the peak bagging analysis of a red giant star in \citet{cor15}. We then compare the set of the obtained frequencies of mixed dipole modes with those from the asymptotic relation proposed by \citet{mos12}, which we compute using different values of $\Delta \Pi_1$. The result shows a significant better matching when values of $\Delta \Pi_1$ around $200 \ \rm{s}$ are used. This confirms that the oscillating star is settled on the core-He-burning phase of stellar evolution.} This result is supported statistically by \citet{mig14}, who report it is more likely to find red clump stars than red giant branch stars in asteroseismic binaries in \emph{Kepler} data. This is largely due to the fact that evolved stars spend more time on the horizontal branch than the red giant branch. We note that due to the large noise level of the mixed modes, we are unable to measure a core rotation rate in the manner of \citet{bec12} and \citet{mos12}. 

\textbf{Need to incorporate the following.}

\textbf{PAUL}

\revise{Due to the noise and damped oscillation, it is difficult to unambiguously derive the mixed mode pattern, described by \citet{mos12}. Therefore we searched for a stars, which PSD resembles the oscillation spectrum of KIC 9246715. A good match was found with the star KIC 11725564 (Figure COMING SOON), resembling well the radial and quadrupole modes, as well as the mixed mode pattern.

Calculating the autocorrelation of the oscillation spectrum of KIC 9246715 from which the radial and quadrupole mode have been pre-whitened and which is converted into period, we find a week, broad peak at about 80 seconds [this is observed $\Delta P_{\rm{obs}}$, not $\Delta \Pi_1$]. A similar result of 87 seconds is found for KIC 11725564, whereby the signal is naturally much cleaner due to the higher mode amplitudes. This period spacing, derived from the autocorrelation corresponds to the observed period spacing, as defined by \citet{bed11} and \citet{mos11} and indicates that the star is indeed likely to be a secondary clump star.}

\textbf{BENOIT}

\revise{The use of the universal red giant oscillation pattern \citep{mos11} for analyzing the spectrum of KIC  9246715 provided the measurement of the large separation $\Delta \nu =  8.33 \pm 0.04 \ \mu\rm{Hz}$, but also exhibited many supernumerary peaks. As soon as it was clear that these peaks cannot be mixed modes, the hypothesis of a binary companion was tested. Again, the universal oscillation pattern allowed us to allocate the supernumerary peaks to a pressure-mode oscillation pattern  based on $\Delta \nu = 8.62 \pm 0.04 \ \mu\rm{Hz}$. The spectra are globally interlaced, with the dipole modes of one component close to the radial modes of the other component, and conversely.

We then measured the asymptotic period spacing with the new method developed by \citet{mos15}. The signature $\Delta \Pi_1 = 150.4 \pm 1.4 \ \rm{s}$ is very clear, despite binarity. In fact, the secondary spectrum cannot mimic a mixed-mode pattern and its global amplitude is small, so that the disturbance is limited. Only one signature is visible, which corresponds to the main component.}
