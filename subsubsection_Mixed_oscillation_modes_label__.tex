\subsubsection{Mixed oscillation modes}
\label{subsubsec_mixed}
Based on the distribution of mixed $\ell = 1$ modes, \citet{gau14} reported that the oscillation pattern period spacing was typical of that of a star from the secondary red clump, i.e., a core-He-burning star that has not experienced a helium flash. This was based on a \revise{dipole gravity mode} period spacing of $\Delta \Pi_1 \simeq 150 \ \rm{s}$. Red giant branch stars have smaller period spacings than red clump stars, and ($\Delta \Pi_1 = 150 \ \rm{s}$, $\Delta \nu = 8.31 \ \mu \rm{Hz}$) puts the oscillating star on the very edge of the asteroseismic parameter space that defines the secondary red clump \citep{mos14}. \revise{Due to noise and damped oscillations, it is difficult to unambiguously determine the mixed mode pattern described by \citet{mos12}. To more accurately assess the evolutionary stage of the oscillating star in KIC 9246715, we employ three different techniques to identify and characterize mixed modes.}

% Enrico
\revise{First, we perform a Bayesian fit to the individual oscillation modes of the star using the \textsc{D\large{iamonds}} code \citep{cor14} and the methodology for the peak bagging analysis of a red giant star in \citet{cor15}. We then compare the set of the obtained frequencies of mixed dipole modes with those from the asymptotic relation proposed by \citet{mos12}, which we compute using different values of $\Delta \Pi_1$. The result shows a significantly better match when values of $\Delta \Pi_1$ around $200 \ \rm{s}$ are used. This confirms that the oscillating star is settled on the core-He-burning phase of stellar evolution.} \newrevise{The results of the \textsc{D\large{iamonds}} fit are in Appendix \ref{appendix}.}

% Paul
\revise{Second, we search for stars with a power density spectrum that resembles the oscillation spectrum of KIC 9246715. As shown in Figure \ref{fig:twin}, a good match is found with the star KIC 11725564, which exhibits very similar radial and quadrupole modes as well as the mixed mode pattern. To find this ``twin,'' we calculate the autocorrelation of the KIC 9246715 oscillation spectrum, pre-whiten its radial and quadrupole modes, and convert it into period. We find a weak, broad peak at about $\Delta P_{\rm{obs}} = 80 \ \rm{s}$. A similar result of $\Delta P_{\rm{obs}} = 87 \ \rm{s}$ is found for KIC 11725564, with a notably cleaner signal thanks to higher mode amplitudes. This corresponds to the observed period spacing as defined by \citet{bed11} and \citet{mos11}, and indicates that the star is indeed likely to be a secondary clump star.}

% Benoit
\revise{Finally, we measure the asymptotic period spacing with the new method developed by \citet{mos15}. The signature $\Delta \Pi_1 = 150.4 \pm 1.4 \ \rm{s}$ is very clear, despite binarity. In fact, the presence of a second oscillation spectrum cannot mimic a mixed-mode pattern because its global amplitude is too small for us to observe a mode disturbance. Only one signature of an oscillating star is visible in a period spacing diagram.}

\revise{We conclude that the mixed oscillation modes in KIC 9246715 are indicative of a secondary red clump star.} This result is supported statistically by \citet{mig14}, who report it is more likely to find red clump stars than red giant branch stars in asteroseismic binaries in \emph{Kepler} data. This is largely due to the fact that evolved stars spend more time on the horizontal branch than the red giant branch. We note that due to the large noise level of the mixed modes, we are unable to measure a core rotation rate in the manner of \citet{bec12} and \citet{mos12}.
