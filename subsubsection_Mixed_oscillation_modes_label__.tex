\subsubsection{Mixed oscillation modes}
\label{subsubsec_mixed}
Based on the distribution of mixed $\ell = 1$ modes, \citet{gau14} reported that the oscillation pattern period spacing was typical of that of a star from the secondary red clump, i.e., a core-He-burning star that has not experienced a helium flash. However, this conclusion was based on a \revise{dipole gravity mode} period spacing of $\Delta \Pi_1 \simeq 150 \ \rm{sec}$, which is difficult to measure robustly from a noisy oscillation spectrum. Red giant branch stars have smaller period spacings than red clump stars, and ($\Delta \Pi_1 = 150 \ \rm{sec}$, $\Delta \nu = 8.31 \ \mu \rm{Hz}$) puts the oscillating star on the very edge of the asteroseismic parameter space that defines the secondary red clump \citep{mos14}.
%Therefore, while asteroseismology does indicate the oscillator in KIC 9246715 is a red clump star, there is a large uncertainty attached to the classification.

\revise{To more accurately assess the evolutionary stage of KIC 9246715, we perform a Bayesian fit to the individual oscillation modes of the star using the \textsc{D\large{iamonds}} code \citep{cor14} and the methodology for the peak bagging analysis of a red giant star in \citet{cor15}. We then compare the set of the obtained frequencies of mixed dipole modes with those from the asymptotic relation proposed by \citet{mos12}, which we compute using different values of $\Delta \Pi_1$. The result shows a significant better matching when values of $\Delta \Pi_1$ around $200 \ \rm{s}$ are used. This confirms that the oscillating star is settled on the core-He-burning phase of stellar evolution.} This result is supported statistically by \citet{mig14}, who report it is more likely to find red clump stars than red giant branch stars in asteroseismic binaries in \emph{Kepler} data. This is largely due to the fact that evolved stars spend more time on the horizontal branch than the red giant branch. We note that due to the large noise level of the mixed modes, we are unable to measure a core rotation rate in the manner of \citet{bec12} and \citet{mos12}. 

\subsubsection{Identifying the oscillating star}\label{identifying}
The asteroseismic mass and surface gravity are consistent with those from the ELC model for both stars, while the asteroseismic radius is only consistent with Star 2. Neither star's mean density agrees with the asteroseismic value, but Star 2 is much closer than Star 1. Overall, our asteroseismic analysis suggests the oscillating star is Star 2. However, we cannot definitely conclude this without considering the temperature dependence of the scaling relations. From \citet{gau13}, \citet{gau14}, and the present work, asteroseismic masses and radii were reported to be $(1.7 \pm 0.3 \ M_\odot, 7.7 \pm 0.4 \ R_\odot)$, $(2.06 \pm 0.13 \ M_\odot, 8.10 \pm 0.18 \ R_\odot)$, and $(2.17 \pm 0.12 \ M_\odot, 8.26 \pm 0.16 \ R_\odot)$, respectively. Among these, $\nu_{\rm{max}}$ does not vary much ($102.2, 106.4, 106.4 \ \mu\rm{Hz}$), and $\Delta \nu$ varies even less ($8.3, 8.32, 8.31 \ \mu\rm{Hz}$), while the assumed temperatures were 4699 K (from the KIC), 4857 K (from \citealt{hub14.2}), and 5000 K (this work).

Even if temperature is the least influential parameter on stellar masses and radii in the asteroseismic scalings, we are at a level of precision where errors on temperature dominate the global asteroseismic results. \revise{Since the scaling equations can be combined to give a mean density that is independent of temperature and $\nu_{\rm max}$ (Eq.~\ref{density}), one might naively expect a good asteroseismic estimation of this quantity compared to the surface gravity.}  While a more in-depth ``peak-bagging'' analysis of individual oscillation modes is beyond the scope of this paper, we strongly suspect the oscillating star is Star 2.

\subsubsection{Surface gravity disagreement}
The asteroseismic $\log g$ measurement nearly agrees with those from ELC, yet all three are some 0.3 dex lower than the spectroscopic $\log g$ values, as can be seen in Table \ref{table2}. This discrepancy is similar to the difference found for giant stars by \citet{hol15}. They investigate a large sample of stars from the ASPCAP (APOGEE Stellar Parameters and Chemical Abundances Pipeline) which have $\log g$ measured via spectroscopy and asteroseismology. They find that spectroscopic surface gravity measurements are roughly 0.2--0.3 dex too high for core-He-burning (red clump) stars and roughly 0.1--0.2 dex too high for shell-H-burning (red giant branch) stars. \citet{hol15} speculate the difference may be partially due to a lack of treatment of stellar rotation, and derive an empirical calibration relation for a ``correct'' $\log g$ for red giant branch stars only. However, the stars in KIC 9246715 do not rotate particularly fast ($v_{\rm{rot}} \sin i \lesssim 8 \ \rm{km \ s}^{-1}$, which includes a contribution from macroturbulence as discussed in Section \ref{parameters}), so we cannot dismiss this discrepancy so readily.