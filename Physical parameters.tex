\section{Physical parameters from light curve \& radial velocities}\label{model}
To derive physical and orbital parameters for KIC 9246715, we use the Eclipsing Light Curve (ELC) code \citep{oro00}. ELC uses a genetic algorithm or Monte Carlo Markov Chain optimizers to simultaneously solve for a suite of stellar parameters. It is able to consider a set of input observables simultaneously, i.e., any combination of light curves and radial velocities. ELC is based on the Wilson-Devinney code \citep{wil71} and uses a full treatment of Roche geometry \citep{avn75}.

To characterize the binary, we compute two sets of ELC models. The first set is done before any constraints from Section \ref{atm} are known. We use the full folded light curve together with all radial velocity points and employ ELC's ``fast analytic mode.'' This uses the equations in \citep{gim06} and treats the two stars as perfect spheres, which is a reasonable assumption for a well-detached binary. We use these preliminary results to guide the spectral disentangling process described in Section \ref{atm}.

The second set of ELC models uses constraints from atmosphere modeling, and breaks the light curves into ``chunks'' to search for stellar activity on the timescale of one orbital period.

IN PROGRESS
% Write more here

In both cases, we use ELC to solve for 16 parameters: orbital period $P_{orb}$, $T_{conj}$ (a zeropoint that sets the deeper primary eclipse to orbital phase $\phi_{ELC} = 0.5$), orbital inclination $i$, $e \sin \omega$ and $e \cos \omega$ (where $e$ is eccentricity and $\omega$ is the longitude of periastron), the temperature of the primary star $T_1$, the mass of the primary star $M_1$, the amplitude of the primary star's radial velocity curve $K_1$, the fractional radii of each star $R_1/a$ and $R_2/a$ (where $a$ is the average orbital seperation), the \emph{Kepler} contamination factor, and stellar limb darkening parameters for the quadratic limb darkening law. Because all binaries obey the relation
\begin{equation}
\frac{M_1}{M_2} = \frac{K_2}{K_1},
\end{equation}
the final result from the ELC model includes masses and radii for both stars.

\begin{itemize}
\item Put a table here with final model values
\item Put a figure here with the final model fit
\end{itemize}
