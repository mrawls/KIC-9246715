\section{Physical parameters from light curve \& radial velocities}\label{model}
To derive physical and orbital parameters for KIC 9246715, we use the Eclipsing Light Curve (ELC) code \citep{oro00}. ELC employs photodynamical modeling with a genetic algorithm or Monte Carlo Markov Chain optimizers to simultaneously solve for a suite of stellar parameters. It is able to consider any set of input constraints simultaneously, i.e., a combination of light curves and radial velocities. ELC is based on the Wilson-Devinney code \citep{wil71} and is able to use a full treatment of Roche geometry \citep{avn75}. It uses the NextGen model atmospheres integrated over a specified filter (in this case, the relatively broad ``white-light'' \emph{Kepler} bandpass).

To characterize the binary, we compute two sets of ELC models. The first set is done before any constraints from Section \ref{atm} are known. We use the full folded light curve together with all radial velocity points and employ ELC's ``fast analytic mode.'' This first set of models uses the equations in \citet{gim06} and treats the two stars as perfect spheres, which is a reasonable assumption for a well-detached binary. We use the parameters from these prelimiary models to guide the spectral disentangling process described in Section \ref{atm}.

The second set of ELC models uses constraints from atmosphere modeling, and breaks the light curves into ``chunks'' to search for stellar activity. One ``chunk'' is a portion of the light curve that includes one primary and one secondary eclipse and has length of order one orbital period (about 171 days for KIC 9246715). This allows us to search for stellar activity that may appear during one eclipse event only. % We keep the spherical assumption?? Or not???

\subsection{Preliminary ELC Model}

We initially use ELC to solve for 16 parameters: orbital period $P_{orb}$, zeropoint $T_{conj}$ (this sets the primary eclipse to orbital phase $\phi_{ELC} = 0.5$ instead of $\phi = 0$), orbital inclination $i$, $e \sin \omega$ and $e \cos \omega$ (where $e$ is eccentricity and $\omega$ is the longitude of periastron), the temperature of the primary star $T_1$, the mass of the primary star $M_1$, the amplitude of the primary star's radial velocity curve $K_1$, the fractional radii of each star $R_1/a$ and $R_2/a$ (where $a$ is the average orbital seperation), the \emph{Kepler} contamination factor, and stellar limb darkening parameters for the quadratic limb darkening law. Because all binaries obey the relation
\begin{equation}
\frac{M_1}{M_2} = \frac{K_2}{K_1},
\end{equation}
the final result from the ELC model includes masses and radii for both stars.

Maybe talk about how we are suspicious of a spot during one of the primary eclipses, which we investigate in the next round of modeling.
