\subsection{Parameters from atmosphere modeling}\label{parameters}
We use the radiative transfer code MOOG \citep{sne73} to estimate $T_{\rm{eff}}$, $\log g$, and metallicity [Fe/H] for the disentangled spectrum of each star in KIC 9246715. First, we use ARES (Automatic Routine for line Equivalent widths in stellar Spectra, \citealt{Sousa_2007}) with a modified {\rm Fe}\kern 0.1em{\sc i} and {\rm Fe}\kern 0.1em{\sc ii} linelist from \citet{tsa13}. ARES automatically measures equivalent widths for spectral lines which can then be used by MOOG. An excellent outline of the process is given by \citet{Sousa_2014}.

We use ARES to identify 66 {\rm Fe}\kern 0.1em{\sc i} and 9 {\rm Fe}\kern 0.1em{\sc ii} lines in the spectrum of Star 1, and 74 {\rm Fe}\kern 0.1em{\sc i} and 10 {\rm Fe}\kern 0.1em{\sc ii} lines in the spectrum of Star 2, all in the 4900--7130 \AA \ region. To arrive at a best-fit stellar atmosphere model with MOOG, we follow the approach of \citet{mag13}. Error bars are determined based on the standard deviation of the derived abundances and the range spanned in excitation potential or equivalent width. For Star 1, we find $T_{\rm{eff}} = 4990 \pm 90 \ \rm{K}$, $\log g = 3.21 \pm 0.45$, and $\rm{[Fe/H]} = -0.22 \pm 0.12$, with a microturbulence velocity of $1.86 \pm 0.09 \ \rm{km \ s}^{-1}$. For Star 2, we find $T_{\rm{eff}} = 5030 \pm 80 \ \rm{K}$, $\log g = 3.33 \pm 0.37$, and $\rm{[Fe/H]} = -0.10 \pm 0.09$, with a microturbulence velocity of $1.44 \pm 0.09 \ \rm{km \ s}^{-1}$.

Projected rotational velocities can also be measured from stellar spectra. To estimate this, we compare the disentangled spectra to a grid of rotationally broadened spectra. We find both stars have $v_{\rm{broad}} \simeq 8 \ \rm{km \ s}^{-1}$. It is important to consider that this observed broadening is a combination of each star's rotational velocity and macroturbulence: $v_{\rm{broad}} = v_{\rm{rot}} \sin i + \zeta_{\rm{RT}}$, where $\zeta_{\rm{RT}}$ is the radial-tangential macroturbulence dispersion \citep{gra78}. We note that rotational broadening is Gaussian while broadening due to macroturbulence is cuspier, but these subtle line profile differences are not distinguishable here. \citet{car08} find a large range of macroturbulence dispersions for giant stars which may vary as a function of luminosity, gravity, and temperature, and introduce a non-physically-motivated empirical relation $v_{\rm{broad}} = [(v_{\rm{rot}} \sin i)^2 + 0.95~\zeta_{\rm{RT}}^2]^{1/2}$, while \citet{tay15} estimate the macroturbulence for giant stars to be on order $10 \%$ of the observed broadening. In any case, at least some of the observed line broadening is attributable to macroturbulence, and we conclude neither star in KIC 9246715 is a particularly fast rotator.

  
  