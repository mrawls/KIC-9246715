\subsection{Stellar parameters}\label{parameters}
We use the radiative transfer code MOOG \citep{sne73} to estimate $T_{\rm{eff}}$, $\log g$, and metallicity [Fe/H] for the disentangled spectrum of each star in KIC 9246715. First, we use ARES (Automatic Routine for line Equivalent widths in stellar Spectra) \citep{Sousa_2007} with a modified Fe I and Fe II linelist from \citet{tsa13}. ARES automatically measures equivalent widths for spectral lines which can then be used by MOOG. An excellent outline of the process is given by \citet{Sousa_2014}.

We use ARES to identify 66 Fe I and 9 Fe II lines in the spectrum of Star 1, and 74 Fe I and 10 Fe II lines in the spectrum of Star 2, all in the 4900--7130 \AA \ region. To arrive at a best-fit stellar atmosphere model with MOOG, we follow the approach of \citet{mag13}. Error bars are determined based on the standard deviation of the derived abundances and the range spanned in excitation potential or equivalent width. For Star 1, we find $T_{\rm{eff}} = 4990 \pm 90 \ \rm{K}$, $\log g = 3.21 \pm 0.45$, and $\rm{[Fe/H]} = -0.22 \pm 0.12$, with a microturbulence velocity of $1.86 \pm 0.09 \ \rm{km \ s}^{-1}$. For Star 2, we find $T_{\rm{eff}} = 5030 \pm 80 \ \rm{K}$, $\log g = 3.33 \pm 0.37$, and $\rm{[Fe/H]} = -0.10 \pm 0.09$, with a microturbulence velocity of $1.44 \pm 0.09 \ \rm{km \ s}^{-1}$. Projected rotational velocities can also be measured from stellar spectra. To estimate this, we compare the disentangled spectra to a grid of rotationally broadened spectra. We find both stars have $v_{\rm{rot}} \sin i \simeq 8 \ \rm{km \ s}^{-1}$. These results are discussed in context in Section \ref{discuss}.

  
  
  
  