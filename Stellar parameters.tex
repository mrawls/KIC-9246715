\subsection{Stellar parameters}\label{parameters}
We use the radiative transfer code MOOG \citep{sne73} to estimate $T_{\rm{eff}}$, $\log g$, and metallicity [Fe/H] for the disentangled spectrum of each star in KIC 9246715. First, we use the Automatic Routine for line Equivalent widths in stellar Spectra \citep[ARES,][]{Sousa_2007} with the Fe I and Fe II linelist from \citet{Yong_2005} UPDATE LINELIST CITATION INFO. ARES automatically measures equivalent widths for spectral lines which can then be used by MOOG. An excellent outline of the process is given by \citet{Sousa_2014}.

We use ARES to identify xx Fe I and xx Fe II lines in the spectrum of Star 1, and xx Fe I and xx Fe II lines in the spectrum of Star 2. SOMETHING ABOUT SIGMA CLIPPING TO SELECT 'GOOD' LINES. To arrive at a robust stellar atmosphere model with MOOG, we begin with INPUT VALUES HERE and follow the approach of \citet{mag13}. Error bars are determined with PROCESS HERE. For Star 1, we find $T_{\rm{eff}} = 5097 \pm 0 \ \rm{K}$, $\log g = 3.14 \pm 0$, and $\rm{[Fe/H]} = 0.02 \pm 0$, with a microturbulence velocity of $1.53 \ \rm{km \ s}^{-1}$. For Star 2, we find $T_{\rm{eff}} = 5061 \pm 0 \ \rm{K}$, $\log g = 3.52 \pm 0$, and $\rm{[Fe/H]} = 0.02 \pm 0$, with a microturbulence velocity of $1.49 \ \rm{km \ s}^{-1}$.

JEAN PLEASE FINISH THIS AND WRITE ABOUT IT! Does it agree with ELC? Why/why not?
    
    
    
    