\textbf{\\ Accepted for publication in The Astrophysical Journal, December 2015 \\ Published February 2016, \href{http://iopscience.iop.org/article/10.3847/0004-637X/818/2/108}{Rawls et al. 2016, ApJ, 818, 108} \\ Read on astro-ph: \url{http://arxiv.org/abs/1601.00038} \\ Blog post summary for the public: \url{http://wp.me/p3H1S0-6X}} \\

\keywords{stars: activity --- binaries: eclipsing --- stars: evolution --- stars: fundamental parameters --- stars: individual (KIC 9246715) --- stars: oscillations}

\section{Introduction}\label{intro}

% EDIT THE COMMENT BELOW TO SPECIFY WHICH FIGURES ARXIVER SHOULD USE
%@arxiver{fig1.pdf,fig4.png,fig15.eps}

Mass and radius are often-elusive stellar properties that are critical to understanding a star's past, present, and future. Eclipsing binaries are the only astrophysical laboratories that allow for a direct measurement of these and other fundamental physical parameters. Recently, however, observing solar-like oscillations in stars with convective envelopes has opened a window to stellar interiors and provided a new way to measure global stellar properties. A pair of asteroseismic scaling relations use the Sun as a \revise{benchmark between} these oscillations and a star's effective temperature to yield mass and radius \citep{kje95,hub10,mos13}.

While both the mass and radius scaling relations are useful, it is important to test their validity. Recent work has investigated the radius relation by comparing the asteroseismic large-frequency separation $\Delta \nu$ and stellar radius between models and simulated data \citep[e.g.][]{ste09,whi11,mig13}, and by comparing asteroseismic radii with independent radius measurements such as interferometry or binary star modeling \citep[e.g.][]{hub11,hub12,sil12}. All of these find that radius estimates from asteroseismology are precise within a few percent, with greater scatter for red giants than main sequence stars. The mass scaling relation remains relatively untested. Most studies test the $\Delta\nu$ scaling with average stellar density and not the scaling of $\nu_{\rm{max}}$ (the asteroseismic frequency of maximum oscillation power) with stellar surface gravity, because the latter has a less-secure theoretical basis \revise{\citep{bel11}.} It is not yet possible to reliably predict oscillation mode amplitudes as a function of frequency \citep{chr12}. One study by \citet{fra13} did test both scaling laws with the red giant eclipsing binary KIC 8410637. They found good agreement between Keplerian and asteroseismic mass and radius, but a more recent analysis from \citet{hub14} indicates that the asteroseismic density of KIC 8410637 is underestimated by $\sim$7\,\% (1.8~$\sigma$, accounting for the density uncertainties), which results in an overestimate of the radius by $\sim$9\,\% (2.7~$\sigma$) and mass by $\sim$17\,\% (1.9~$\sigma$). Additional benchmarks for the asteroseismic scaling relations are clearly needed.

Evolved red giants are straightforward to characterize through pressure-mode solar-like oscillations in their convective zones, and red giant asteroseismology is quickly becoming an important tool to study stellar populations throughout the Milky Way \citep[for a review of this topic, see][]{cha13}. Compared to main-sequence stars, red giants oscillate with larger amplitudes and longer periods---several hours to days instead of minutes. Oscillations appear as spikes in the amplitude spectrum of a light curve that is sampled both frequently enough and for a sufficiently long duration. Therefore, observations from the \emph{Kepler} space telescope taken every 29.4 minutes (long-cadence) over many 90-day quarters are ideal for asteroseismic studies of red giant stars.

\emph{Kepler}'s primary science goal is to find Earth-like exoplanets orbiting sun-like stars \citep{bor10}. However, in addition to successes in planet-hunting and suitability for red giant asteroseismology, \emph{Kepler} is also incredibly useful for studies of eclipsing binary stars. \emph{Kepler} has discovered numerous long-period eclipsing systems from consistent target monitoring over several years \citep{prs11,sla11}. Eclipsing binaries are important tools for understanding fundamental stellar properties, testing stellar evolutionary models, and determining distances. When radial velocity curves exist for both stars in an eclipsing binary, along with a well-sampled light curve, the inclination is precisely constrained and a full orbital solution with masses and radii can be found. Kepler's third law applied in this way is the \emph{only} direct method for measuring stellar masses.

Taken together, red giants in eclipsing binaries (hereafter RG/EBs) that exhibit solar-like oscillations are ideal testbeds for asteroseismology. There are presently 18 known RG/EBs that show solar-like oscillations \citep{hek10,gau13,gau14,bec14,bec15} with orbital periods ranging from 19 to 1058 days, all in the \emph{Kepler} field of view.

In this paper, we present physical parameters for the unique RG/EB KIC 9246715 with a combination of dynamical modeling, stellar atmosphere modeling, and asteroseismology. KIC 9246715 contains two nearly-identical red giants in a 171-day eccentric orbit with a single main set of solar-like oscillations. A second set of oscillations, \newrevise{potentially} attributable to the other star, is marginally detected. The system's derived physical parameters are in agreement with \citet{hel15}, which was prepared simultaneously and independently. In \S \ref{data}, we describe how we acquire and process photometric and spectroscopic data, and \S \ref{rvs} explains our radial velocity extraction process. In \S \ref{atm}, we disentangle each star's contribution to the spectra to perform stellar atmosphere modeling. We then present our final orbital solution and physical parameters for KIC 9246715 in \S \ref{model}. Finally, \S \ref{discuss} compares our results with global asteroseismology and discusses the connection among solar-like oscillations, stellar evolution, and effects such as star spots and tidal forces, as well as implications for future RG/EB studies.

  
  
  