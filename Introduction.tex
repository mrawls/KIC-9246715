\section{Introduction}\label{intro}

% big picture context
Mass and radius are often-elusive stellar properties that are critical to understanding a star's past, present, and future. Eclipsing binaries are the only astrophysical laboratories that allow for a direct measurement of these and other fundamental physical parameters. Recently, however, observing solar-like oscillations in stars with convective envelopes has opened a window to stellar interiors and provided a new way to measure global stellar properties. Asteroseismic scaling relations are an empirical connection between these oscillations, effective temperature, mass, and radius \citep{kje95,hub10,mos13}. While these relations are useful, they remain relatively untested. One notable exception is the red giant in the eclipsing binary KIC 8410637, which shows good agreement between Keplerian and asteroseismic mass and radius \citep{fra13}. In this work, we consider another archetypal case: KIC 9246715, an eclipsing binary composed of one oscillating red giant and one non-oscillating red giant.

% asteroseismology w/Kepler is an awesome tool
Stars with convective outer layers potentially exhibit solar-like oscillations. These oscillations depend on the physical processes in their interiors. In particular, evolved red giants are increasingly easy to characterize through these pressure-mode oscillations \citep[for a review of this topic, see][]{cha13}. Compared to main-sequence stars, red giants oscillate with larger amplitudes and longer periods---several hours to days instead of minutes. Oscillations appear as spikes in the amplitude spectrum of a light curve that is sampled both frequently enough and for a sufficiently long duration. Given these two requirements, observations from the \emph{Kepler} space telescope taken every 29.4 minutes (long-cadence) over many 90-day quarters are ideal for asteroseismic studies of red giant stars.

% EBs are really awesome w/Kepler too
\emph{Kepler}'s primary science goal is to find Earth-like exoplanets orbiting sun-like stars \citep{bor10}. However, in addition to successes in planet-hunting and suitability for red giant asteroseismology, \emph{Kepler} is also incredibly useful for studies of eclipsing binary stars. In fact, \emph{Kepler} has discovered numerous long-period eclipsing systems from consistent target monitoring over several years. Eclipsing binaries are extremely important tools for understanding fundamental stellar properties, and in turn for testing stellar evolutionary models or determining distances. When radial velocity curves exist for both stars in an eclipsing binary, along with a well-sampled light curve, a full orbital solution can be found. Accurate masses and radii are straightforward to derive from such a solution; indeed, Kepler's third law applied in this way is the \emph{only} direct method for measuring stellar masses.

Taken together, red giants in eclipsing binaries (hereafter RG/EBs) that exhibit solar-like oscillations are an ideal testbed for asteroseismology. There are presently 15 known RG/EBs that show solar-like oscillations \citep{gau13,gau14}. All of these have orbital periods ranging from tens to hundreds of days, and are found in the \emph{Kepler} field of view.

% Overview of paper
In this paper, we present physical parameters for the unique RG/EB KIC 9246715, which contains two nearly-identical red giants in a 171-day eccentric orbit. Only one set of solar-like oscillations is present, and the oscillations show a lower amplitude than similar giants. We find good agreement between photodynamic models and asteroseismology for the oscillating star's mass and radius, but are unable to definitively say which star is oscillating.
%We find $M_1 = 2.16 \pm 0.04\ M_{\odot}$ and $M_2 = 2.14 \pm 0.03\ M_{\odot}$ (compare with $M_{\rm{seismo}} = 2.06 \pm 0.13 \ M_{\odot}$), and $R_1 = 7.90 \pm 0.04 \ R_{\odot}$ and $R_2 = 8.33 \pm 0.04 \ R_{\odot}$ (compare with $R_{\rm{seismo}} = 8.10 \pm 0.18 \ R_{\odot}$).
We explore how star spots and tidal forces may influence the oscillations, verify that the two stars are consistent with a co-evolutionary history, and discuss how this system can inform future detailed studies of RG/EBs.

In \S \ref{data}, we describe how we acquire and process photometric and spectroscopic data, and \S \ref{rvs} explains our radial velocity extraction process. In \S \ref{atm}, we disentangle each star's contribution to the spectra to perform stellar atmosphere modeling. We then present our final orbital solution and physical parameters for KIC 9246715 in \S \ref{model}. Finally, \S \ref{discuss} compares our results with those from asteroseismology and discusses the connection between solar-like oscillations, effects such as star spots and tidal forces, and implications for future RG/EB studies.
