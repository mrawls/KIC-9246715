\section{Conclusions}\label{conclude}

\revise{REVISIONS IN PROGRESS}

We have characterized the double red giant eclipsing binary KIC 9246715 with a combination of dynamical modeling, stellar atmosphere modeling, and global asteroseismology, and have investigated the roles of magnetic activity, tidal forces, and stellar evolution in creating the system we observe today. KIC 9246715 represents a likely future state of similar-mass RG/EB systems and raises interesting questions about the interactions among stellar activity, tides, and solar-like oscillations.

The two stars in KIC 9246715 are nearly twins ($M_1 = 2.171\substack{+0.006 \\ -0.008} \ M_{\odot}$, $M_2 = 2.149\substack{+0.006 \\ -0.008} \ M_{\odot}$, $R_1 = 8.37\substack{+0.03 \\ -0.07} \ R_{\odot}$, $R_2 = 8.30\substack{+0.04 \\ -0.03} \ R_{\odot}$), yet we find only one set of solar-like oscillations strong enough to measure robustly ($M = 2.17 \pm 0.12 \ M_{\odot}$, $R = 8.26 \pm 0.16 \ R_{\odot}$). These oscillations are most likely from Star 2, the larger and cooler of the pair, which appears to be less magnetically active than Star 1. We identify a second set of marginally detectable oscillations attributable to Star 1, for which only $\Delta \nu$ can be estimated, yielding a higher average density than the main oscillation spectrum.

Surface gravities from dynamical modeling ($\log g_1 = 2.929\substack{+0.007 \\ -0.003}$, $\log g_2 = 2.932\substack{+0.003 \\ -0.004}$) and asteroseismology ($\log g = 2.942 \pm 0.007$) nearly agree, while the surface gravity from stellar atmosphere modeling is higher ($\log g_1 = 3.21 \pm 0.45$, $\log g_2 = 3.33 \pm 0.37$). A similar discrepancy has been found between the asteroseismic and spectroscopic surface gravities of many giant stars, but the physical cause is unknown. The present evolutionary state of the stars in KIC 9246715 is unclear. Radii from a stellar evolution model suggest that both stars should be on the red giant branch with an age of $8.6 \times 10^8$ years, while the oscillation mode period spacing---which is difficult to measure due to a weak and noisy oscillation spectrum---indicates the main oscillating star is on the red clump.

Future work will use accurate ``peak bagging'' of individual oscillation modes with Bayesian methods to disentangle the oscillation spectra of both stars. This will allow inversion of the measured frequencies to make an asteroseismic model that can be compared with other oscillating giants. It will also nail down the evolutionary state, constrain convective overshoot, and inform how we classify the evolutionary state of other RG/EBs.

Red giants are ideal tools for probing the Milky Way Galaxy via asteroseismology, so it is crucial that we understand the accuracy and precision of asteroseismically-derived physical parameters. Along the same lines, more than half of cool stars should be in binary or multiple systems, so galactic studies must be done carefully due to external influences of binarity on solar-like oscillations. Detailed studies of the handful of known RG/EBs are crucial to ensure we understand these galactic beacons. Future work will characterize the other known oscillating RG/EBs as well as several non-oscillating RG/EBs. These have the potential to become some of the best-studied stars while simultaneously helping us better understand the structure of the Milky Way.
