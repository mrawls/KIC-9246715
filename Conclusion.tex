\section{Conclusions}\label{conclude}

We have characterized the double red giant eclipsing binary KIC 9246715 with a combination of dynamical modeling, stellar atmosphere modeling, and global asteroseismology, and have investigated the roles of magnetic activity, tidal forces, and stellar evolution in creating the system we observe today. KIC 9246715 represents a likely future state of similar-mass RG/EB systems and raises interesting questions about the interactions among stellar activity, tides, and solar-like oscillations.

The two stars in KIC 9246715 are nearly twins ($M_1 = 2.171\substack{+0.006 \\ -0.008} \ M_{\odot}$, $M_2 = 2.149\substack{+0.006 \\ -0.008} \ M_{\odot}$, $R_1 = 8.37\substack{+0.03 \\ -0.07} \ R_{\odot}$, $R_2 = 8.30\substack{+0.04 \\ -0.03} \ R_{\odot}$), yet we find only one set of solar-like oscillations strong enough to measure robustly ($M = 2.17 \pm 0.12 \ M_{\odot}$, $R = 8.26 \pm 0.16 \ R_{\odot}$). \revise{While the asteroseismic mass and radius agree with both Star 1 and Star 2, the surface gravity and average density derived from asteroseismology ($\log g = 2.942 \pm 0.007$, $\bar{\rho}_{\rm{seismic}}/\bar{\rho}_\odot = (3.862 \pm 0.009) \times 10^{-3}$) are a closer match for Star 2 (compare $\log g_1 = 2.929\substack{+0.007 \\ -0.003}$, $\log g_2 = 2.932\substack{+0.003 \\ -0.004}$; $\bar{\rho}_1/\bar{\rho}_\odot = (3.70\substack{+0.10 \\ -0.04}) \times 10^{-3}$, $\bar{\rho}_2/\bar{\rho}_\odot = (3.76\substack{+0.04 \\ -0.05}) \times 10^{-3}$). We therefore conclude that the main} oscillations are most likely from Star 2, which appears to be less magnetically active than Star 1. We identify a second set of marginally detectable oscillations potentially attributable to Star 1, for which only $\Delta \nu$ can be estimated, yielding a higher average density than the main oscillation spectrum. \revise{This is not consistent with the expected density of Star 1, however, which should be less than that of Star 2.}

Surface gravities from dynamical modeling and asteroseismology nearly agree, while surface gravities from stellar atmosphere modeling are higher ($\log g_1 = 3.21 \pm 0.45$, $\log g_2 = 3.33 \pm 0.37$). A similar discrepancy has been found between the asteroseismic and spectroscopic surface gravities of other giant stars, but the physical cause is unknown. \revise{Radii from stellar evolution models are consisted with a pair of nearly-coeval stars either on the red giant branch with an age of approximately $8.3 \times 10^8$ years, or coeval stars on the horizontal branch with an age of about $9.3 \times 10^8$ years. However, the period spacing of mixed oscillation modes clearly indicates that the main oscillator in KIC 9246715 is on the secondary red clump, and we conclude that KIC 9246715 is a pair of secondary red clump stars.}

Red giants are ideal tools for probing the Milky Way Galaxy via asteroseismology, so it is crucial that we understand the accuracy and precision of asteroseismically-derived physical parameters. Along the same lines, more than half of cool stars should be in binary or multiple systems, so galactic studies must be done carefully due to external influences of binarity on solar-like oscillations. Detailed studies of the handful of known RG/EBs are crucial to ensure we understand these galactic beacons. Future work will characterize the other known oscillating RG/EBs as well as several non-oscillating RG/EBs. These have the potential to become some of the best-studied stars while simultaneously helping us better understand the structure of the Milky Way.
