\section{Conclusions}\label{conclude}
Red giants are ideal beacons for probing the Milky Way Galaxy via asteroseismology. However, more than half of cool stars should be in binary or multiple systems, so galactic studies must be done carefully due to external influences of binarity on solar-like oscillations. Detailed studies of the handful of known RG/EBs are crucial to ensure we understand the physical properties of these galactic beacons. In fact, RG/EBs are the only stars that enable us to measure their global physical properties with two independent methods.

To this end, the double red giant eclipsing binary KIC 9246715 is a useful case study. The two stars are nearly twins ($M_1 = 2.16 \pm 0.04\ M_{\odot}$, $M_2 = 2.14 \pm 0.03\ M_{\odot}$, $R_1 = 7.90 \pm 0.04 \ R_{\odot}$, $R_2 = 8.33 \pm 0.04 \ R_{\odot}$), yet we observe only one pair of solar-like oscillations. The oscillations are most likely from Star 2, the slightly larger and cooler of the pair.

WRITE MORE WORDS HERE.

This supports the idea of magnetically-active stars having low-amplitude solar-like oscillations or none at all.

Our detailed light curve and radial velocity modeling... 

The fact that only one star is oscillating points to how tidal activity may affect oscillations.


  