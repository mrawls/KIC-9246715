\section{Conclusions}\label{conclude}
We characterize the double red giant eclipsing binary KIC 9246715 with a combination of light curve + radial velocity modeling, stellar atmosphere modeling, and global asteroseismology, and investigate the role magnetic activity, tidal forces, and stellar evolution have had in creating the system we observe today. KIC 9246715 represents a likely future state of similar-mass RG/EB systems and raises interesting questions about the interactions among magnetic activity, tides, and solar-like oscillations.

The two stars in KIC 9246715 are nearly twins ($M_1 = 2.16 \pm 0.04\ M_{\odot}$, $M_2 = 2.14 \pm 0.03\ M_{\odot}$, $R_1 = 7.90 \pm 0.04 \ R_{\odot}$, $R_2 = 8.33 \pm 0.04 \ R_{\odot}$), yet we observe only one pair of solar-like oscillations ($M = 2.17 \pm 0.13 \ M_{\odot}$, $R = 8.26 \pm 0.16 \ R_{\odot}$). The oscillations are therefore most likely from Star 2, the larger and cooler of the pair, which may also be less magnetically active than Star 1. Surface gravity measurements from photodynamic modeling and asteroseismology nearly agree (x and x, respectively), while the measured surface gravity from stellar atmosphere modeling is significantly higher (x). A discrepancy similar to this is found in a population study comparing asteroseismic and spectroscopic surface gravities \citep{hol15}, but the physical cause is uncertain.

The stars were born and evolved together, but the present evolutionary state (red giant branch vs. red clump) is unclear. Tidal circularization timescales and radii from a stellar evolution model indicate that both stars should be on the red giant branch with an age of $7.28 \times 10^8$ years, while the oscillation mode period spacing---which is difficult to measure due to a weak and noisy oscillation spectrum---suggests the oscillating star may be on the red clump. Future work will examine this disagreement in more detail with precise ``peak-bagging'' of individual oscillation modes.

Red giants are ideal beacons for probing the Milky Way Galaxy via asteroseismology, so it is crucial that we understand the accuracy and precision of asteroseismically-derived physical parameters. Along the same lines, more than half of cool stars should be in binary or multiple systems, so galactic studies must be done carefully due to external influences of binarity on solar-like oscillations. Detailed studies of the handful of known RG/EBs are crucial to ensure we understand these galactic beacons. Future work will characterize the other 14 known oscillating RG/EBs as well as several non-oscillating RG/EBs. These have the potential to become some of the best-studied stars while simultaneously helping us better understand the structure of the Milky Way.
  