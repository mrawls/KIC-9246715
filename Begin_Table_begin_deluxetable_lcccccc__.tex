% Begin Table
\begin{deluxetable}{lcccccc}
\tablecolumns{7}
%\tablewidth{0pt}
\tabletypesize{\small}
\tablecaption{Physical parameter comparisons for KIC 9246715 with different modeling techniques.}
\centering
\tablehead{
\colhead{} \vspace{-0.15cm} & \colhead{Mass} & \colhead{Radius} & \colhead{$\log g$} & \colhead{$T_{\rm{eff}}$} & & \colhead{Age} \\ 
\colhead{Model}  \vspace{-0.15cm} & & & & & \colhead{[Fe/H]} & \\
\colhead{} & \colhead{($M_{\odot}$)} & \colhead{($R_{\odot}$)} & \colhead{(cgs)} & \colhead{(K)} &  & \colhead{(Gyr)}
}
\startdata
ELC (Light Curve + RV)		&	$1.1 \pm 0.1$	& 	$1.1 \pm 0.1$	&	$1.1 \pm 0.1^{\dagger}$	&	\nodata	&	\nodata	&	\nodata 	\\
Global Asteroseismology		&	$1.1 \pm 0.1$	& 	$1.1 \pm 0.1$	&	$1.1 \pm 0.1^{\dagger}$	&	\nodata	&	\nodata	&	\nodata 	\\
Stellar Atmosphere (Spectroscopy)	&	\nodata	& 	\nodata	 & 	$1.1 \pm 0.1$	&	$1.1 \pm 0.1$	&	$1.1 \pm 0.1$	&	\nodata
\enddata
\label{table2}
\tablenotetext{$\dagger$}{These values of $\log g$ are measured indirectly from the mass and radius.}
\end{deluxetable}
% End Table
    