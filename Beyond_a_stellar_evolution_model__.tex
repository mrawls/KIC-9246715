Beyond a stellar evolution model, it is important to consider how each star has affected the other over time. To estimate how tidal forces change orbital eccentricity, we follow the approach of \citet{ver95}. They use a theory of the equilibrium tide first proposed by \citet{zah77} to calculate a timescale for orbit circularization as a star evolves. It is important to note that \citet{ver95} assumed circularization would proceed by a small secondary star (main sequence or white dwarf) imposing an equilibrium tide on a large giant, while the situation with KIC 9246715 is more complicated. For a thorough review of tidal forces in stars, see \citet{ogi14}.

The timescale $\tau_c$ on which orbital circularization occurs is given by $1/\tau_c \equiv \rm{d} \ln e / \rm{d}t$ \citep{ver95}. To estimate the change in eccentricity over the lifetime of KIC 9246715, we integrate over the orbit circularization timescale:

\begin{equation}\label{circ}
\Delta \ln e = \int_0^t \frac{\rm{d}t'}{\tau_c(t')}.
\end{equation}

We assume the stars are sufficiently separated to make $a$ constant over time, and we further assume no mass loss. We can then use Equations 5 and 6 from \citet{ver95} to calculate the expected change in orbital eccentricity:

\begin{equation}\label{tide1}
\Delta \ln e = -1.7 \times 10^{-5} \ f \ {\left( \frac{M}{M_{\odot}} \right)}^{-11/3} \ q(1+q)^{-5/3} \ I(t) {\left( \frac{P_{\rm{orb}}}{\rm{day}} \right)}^{-16/3},
\end{equation}

where $f$ is a factor of order unity, $q$ is the mass ratio, and

\begin{equation}
I(t) \equiv \int_0^t \left( \frac{T_{\rm{eff}}(t')}{4500 \rm{K}} \right)^{4/3} \ \left( \frac{M_{\rm{env}}(t')}{M_{\odot}} \right)^{2/3} \ \left( \frac{R(t')}{R_{\odot}} \right)^8 \ dt',
\end{equation}

where $M_{\rm{env}}$ is the mass of the convective envelope of the primary star.

For the MESA model described above with $M = 2.16 \ M_{\odot}$, we compute $\Delta \ln e = -3.04 \times 10^{-5}$ up until $t = 7.28 \times 10^8$ years (the age corresponding to $R \simeq 8 \ R_{\odot}$). Rewriting this as $\log [-\Delta \ln e] = -4.52$, a value clearly less than zero, indicates that the binary has \emph{not} had sufficient time to circularize its orbit, though it is possible the system's initial eccentricity was higher than the $e = 0.35$ we observe today.

The two coeval stars in KIC 9246715 have very similar masses, radii, and temperatures, so this rough calculation is valid both for Star 1 acting on Star 2 and vice versa. In contrast, given just another $0.4 \times 10^8$ years to evolve past the red giant branch toward the red clump, $\log [-\Delta \ln e]$ becomes greater than zero and the expectation is a circular orbit. Therefore, the observed $e > 0$ is consistent only with a younger red giant branch star, and not with an older red clump star.

Tidal forces also tend to synchronize a binary star's orbit with the stellar rotation period, generally on shorter timescales than required for circularization \citep{ogi14}. Hints of KIC 9246715's stellar rotation behavior are present throughout this study: quasi-periodic light curve variability on the order of half the orbital period (Section \ref{discuss}), a star spot present during one primary eclipse event only (Section \ref{segment}), a constraint on $v \sin i$ from spectra (Section \ref{parameters}), and asteroseismic period spacing consistent with red clump stars yet not clear enough to measure a robust core rotation rate (Section \ref{discuss}). While full tidal circularization has not occurred, it is clear that modest tidal forces have played a role in the evolution of KIC 9246715, and may be linked to the absence or weakness of solar-like oscillations. Future studies of RG/EBs with different evolutionary histories and orbital configurations will help explore this connection further.
  
  
  