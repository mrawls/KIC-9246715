Beyond a stellar evolution model, it is important to consider how each star has affected the other over time. When the two stars in KIC 9246715 reach the tip of the red giant branch, they have radii of approximately $25 \ R_\odot$, which is still significantly smaller than the periastron separation ($r_{\rm{peri}} = (1-e)~a = 137 \ R_\odot$). We never expect the stars to experience a common envelope phase, so this cannot be used to constrain the present evolutionary state.

To estimate how tidal forces change orbital eccentricity, we follow the approach of \citet{ver95}. They use a theory of the equilibrium tide first proposed by \citet{zah77} to calculate a timescale for orbit circularization as a star evolves. It is important to note that \citet{ver95} assumed circularization would proceed by a small secondary star (main sequence or white dwarf) imposing an equilibrium tide on a large giant, while the situation with KIC 9246715 is more complicated. For a thorough review of tidal forces in stars, see \citet{ogi14}.

From Equation 2 in \citet{ver95}, the timescale $\tau_c$ on which orbital circularization occurs is given by

\begin{eqnarray}
\frac{1}{\tau_c} & \equiv &
\frac{{\rm{d}} \ln e} {{\rm{d}} t} \nonumber \\
& \simeq & -1.7 {\left( \frac{T_{\rm{eff}}}{4500 \rm{K}} \right)}^{4/3} \left( \frac{M_{\rm{env}}}{M_{\odot}} \right)^{2/3} \\
& \times & \ \frac{M_{\odot}}{M} \frac{M_2}{M} \frac{M+M_2}{M} \left( \frac{R}{a} \right)^8 \ \rm{yr}^{-1}, \nonumber
\end{eqnarray}

\noindent where $M$, $R$, and $T_{\rm{eff}}$ are the mass, radius, and temperature of a giant star with dissipative tides, $M_{\rm{env}}$ is the mass of its convective envelope, $M_2$ is the mass of the companion star, and $a$ is the semi-major axis of the binary orbit.

We integrate this expression over the lifetime of KIC 9246715 to estimate the total expected change in orbital eccentricity, $\Delta \ln e$. We assume $a$ is constant and that there is no mass loss. Because KIC 9246715 is a detached binary, we can separate the integral into a part that is independent of the orbit and a part that must be integrated over time:

\begin{eqnarray}\label{tide1}
\Delta \ln e  & = &
\int_0^t \frac{\rm{d}t'}{\tau_c(t')} \nonumber \\
& \simeq & -1.7 \times 10^{-5} {\left( \frac{M}{M_{\odot}} \right)}^{-11/3} \\
& \times & \ q(1+q)^{-5/3} \ I(t) {\left( \frac{P_{\rm{orb}}}{\rm{day}} \right)}^{-16/3}, \nonumber
\end{eqnarray}

\noindent where $q$ is the mass ratio and

\begin{equation}
I(t) \equiv \int_0^t \left( \frac{T_{\rm{eff}}(t')}{4500 \rm{K}} \right)^{4/3} \left( \frac{M_{\rm{env}}(t')}{M_{\odot}} \right)^{2/3} \left( \frac{R(t')}{R_{\odot}} \right)^8 dt'. \nonumber
\end{equation}

For the MESA model described above with $M = 2.15 \ M_{\odot}$, \revise{we compute $\Delta \ln e = -x.x \times 10^{-5}$ up until $t = 8.3 \times 10^8$ and $\Delta \ln e = -x.x \times 10^{-5}$ up until $t = 9.3 \times 10^8$ years (the ages corresponding to $R \simeq 8.3 \ R_{\odot}$). Rewriting these as $\log [-\Delta \ln e] = -x.x$ and $\log [-\Delta \ln e] = -x.x$, which are both less than zero}, indicates that the binary has \emph{not} had sufficient time to circularize its orbit, though it is possible the system's initial eccentricity was higher than the $e = 0.35$ we observe today.

The two stars in KIC 9246715 have very similar masses, radii, and temperatures, so this rough calculation is valid both for Star 1 acting on Star 2 and vice versa. Given more time to evolve past the tip of the red giant branch and well onto the red clump (with $R \simeq 25 \ R_\odot$ for the second time), $\log [-\Delta \ln e]$ becomes greater than zero and the expectation is a circular orbit. Therefore, the observed eccentricity is consistent with \revise{both a red giant branch star aged approximately $8.3 \times 10^8$ years and with a secondary red clump star just past the tip of the red giant branch aged approximately $9.3 \times 10^8$ years.}

Tidal forces also tend to synchronize a binary star's orbit with the stellar rotation period, generally on shorter timescales than required for circularization \citep{ogi14}. Hints of KIC 9246715's stellar rotation behavior are present throughout this study: quasi-periodic light curve variability on the order of half the orbital period, \revise{residual scatter between light curve observations and the best-fit model during both eclipses}, a constraint on $v_{\rm{rot}} \sin i$ from spectra, and an asteroseismic period spacing consistent with a red clump star yet not clear enough to measure a robust core rotation rate.

While full tidal circularization has not occurred, it is clear that modest tidal forces have played a role in the evolution of KIC 9246715, and may be linked to the absence or weakness of solar-like oscillations. Future studies of RG/EBs with different evolutionary histories and orbital configurations will help explore this connection further.

  
  