\section{Observations}\label{data}

\subsection{\emph{Kepler} light curves}\label{kepler}
Our light curves are from the \emph{Kepler} satellite in long-cadence mode (one data point every 29.4 minutes), and span 17 quarters---roughly four years---with only occasional gaps. Long-cadence observations are well-suited for red giant asteroseismology, as main sequence stars oscillate too rapidly to be seen. In addition, consistent target monitoring over several years has allowed \emph{Kepler} to discover numerous long-period eclipsing systems.

When studying long-period eclipsing binaries, it is important to remove instrumental effects in the light curve while preserving the astrophysically interesting signal. In this work, we prioritize preserving eclipses and any stellar variability therein. Our minimalistic detrending algorithm uses the simple aperture photometry (SAP) long-cadence \emph{Kepler} data for quarters 0--17. First, any observations with NaNs are removed. Next, observations from different quarters are put onto the same median level, and the ends of each quarter are adjusted to remove discontinuities. We assume the overall system flux does not change over a timescale of years, and remove any long-term trend by fitting a third-order polynomial to the entire light curve. Finally, we keep only the portions of the light curve that lie within one eclipse duration of the start and end of each eclipse. This differs from the light curve processing needed for asteroseismology, which requires no gaps in time and typically ``fills'' the eclipses so they do not dominate the power spectrum \citep{gau14}.

The result is shown in Figure \ref{fig:keplerfig}. The top panel shows the full detrended light curve with some instrumental effects still present, while the middle and bottom panels indicate the regions near each eclipse used in this work. We adopt the convention that the ``primary'' eclipse is the deeper of the two. In this case, primary eclipse corresponds to when the larger star passes in front of its smaller companion.
