\section{Observations}\label{data}

\subsection{\emph{Kepler} light curves}\label{kepler}
Our light curves are from the \emph{Kepler} satellite in long-cadence mode (one data point every 29.4 minutes), and span 17 quarters---roughly four years---with only occasional gaps. Long-cadence observations are well-suited for red giant asteroseismology, as main sequence stars oscillate too rapidly to be seen. In addition, consistent target monitoring over several years has allowed \emph{Kepler} to discover numerous long-period eclipsing systems.

When studying long-period eclipsing binaries, it is important to remove instrumental effects in the light curve while preserving the astrophysically interesting signal. In this work, we prioritize eclipses and stellar variability. Our minimalistic detrending algorithm uses the simple aperture photometry (SAP) long-cadence \emph{Kepler} data for quarters 0--17. First, any observations with NaNs are removed. Next, observations from different quarters are put onto the same median level, and the ends of each quarter are adjusted to remove discontinuities. Finally, we assume the overall system flux does not change over a timescale of years, and remove any long-term trend by fitting a third-order polynomial to the entire light curve. The result is shown in Figure \ref{fig:keplerfig}. This differs from the light curve processing in \citet{gau14}, which required interpolation and maually removing the eclipses to extract the oscillation frequencies. In this work, we adopt the convention that the ``primary'' eclipse is the deeper of the two. In this case, primary eclipse corresponds to when the larger, brighter star passes in front of its smaller, dimmer companion.
