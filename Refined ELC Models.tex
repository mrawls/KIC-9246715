\subsection{Light curve segment ELC models}\label{segment}
To investigate secular changes in KIC 9246715, we split the \emph{Kepler} light curve into seven segments such that each contains one primary and one secondary eclipse. This was particularly motivated by the residuals of the primary eclipse in the all-eclipse model, as shown in Figure \ref{fig:ELCresult}. Of the eight observed primary eclipses, one is slightly shallower than the others by about 0.004 magnitudes. This indicates that Star 1 was slightly dimmer during this primary eclipse than the other seven, which we attribute to star spot activity.

We therefore calculate a second set of parameters based on the root-mean-square (RMS) of seven ELC models, one for each light curve segment. Each segment still includes the full set of radial velocity data. The values reported are the RMS of these seven models, $a_{\rm{RMS}} = \sqrt{\frac{1}{n} \sum_{i=1}^n (a_i^2)}$, plus or minus the RMS error, $\sqrt{\frac{1}{n} \sum_{i=1}^n (a_i - a_{\rm{RMS}})^2}$. These are in the ``LC segment RMS'' column of Table \ref{table1}. As expected, the RMS error for the radius of Star 1 is an order of magnitude larger than in the all-eclipse model. (The error bars for on masses, on the other hand, are unrealistically small because the radial velocity data constrains component masses and that input is the same for all seven models.) The ELC model for the light curve segment with the notably shallower primary eclipse returns $R_1 = 8.13 \ R_{\odot}$. We emphasize that the radius is not actually changing, but that ELC is assuming a dimmer Star 1 must be larger to produce the same total flux. Detailed spot modeling is beyond the scope of this paper.

Because there are only seven pairs of eclipses and one is an outlier, we adopt the all-eclipse ELC solution in this work, with the exception of temperature. The white-light \emph{Kepler} bandpass is not well-suited to constrain stellar temperatures, so in this situation ELC is calculating a temperature based on other, better-constrained model parameters. We therefore adopt the more conservative RMS temperature error bars $(\pm ~200 \ \rm{K})$ together with the all-eclipse ELC temperature value. For all parameters, both the all-eclipse model and the RMS results agree within their respective errors.
