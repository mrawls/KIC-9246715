\subsection{Light curve segment ELC models}\label{segment}
To investigate secular changes in KIC 9246715, we split the \emph{Kepler} light curve into seven segments such that each contains one primary and one secondary eclipse. This is particularly motivated by \revise{the photometric variability seen in Figure \ref{fig:lcfig2} and} the residuals of the primary eclipse in the all-eclipse model, as shown in Figure \ref{fig:ELCresult}. Of all the observed primary eclipses, the one in the seventh light curve segment is slightly shallower than the others by about 0.004 magnitudes. \revise{To learn why, we examine the \emph{Kepler} Target Pixel Files, which reveal the aperture used for KIC 9246715 includes a larger portion of a nearby contaminating star every fourth quarter. This higher contamination is coincident with the secondary eclipse only in the fifth light curve segment and both eclipses in the seventh light curve segment. Higher contamination results in shallower eclipses because there is an overall increase in flux, and we conclude that the shallower primary eclipse is a result of this contamination rather than a star spot or other astrophysical signal.}

We therefore calculate a second set of parameters based on the root-mean-square (RMS) of \revise{six ELC models, one for each light curve segment, excluding the seventh segment which has significantly higher contamination in both eclipses}. Each segment still includes the full set of radial velocity data. The values reported are the RMS of these seven models, $a_{\rm{RMS}} = \sqrt{\frac{1}{n} \sum_{i=1}^n (a_i^2)}$, plus or minus the RMS error, $\sqrt{\frac{1}{n} \sum_{i=1}^n (a_i - a_{\rm{RMS}})^2}$. These are reported in Table \ref{table1}. \revise{Temperature is excluded because the white-light \emph{Kepler} bandpass is not well-suited to constrain stellar temperatures, and the RMS errors among each light curve segment are artificially small.}

For all parameters, the all-eclipse model and the LC segment model agree \revise{within $2\sigma$. We note that $\omega$, the \emph{Kepler} contamination, and $R_1$ all have significantly larger error bars in the LC segment results than the all-eclipse results. This reflects an inherent degeneracy between viewing angle and stellar radius in a binary with grazing eclipses, which is exacerbated by uncertainties in limb darkening and temperature, as well as varying contamination between quarters. When we hold both stars' limb darkening fixed with theoretical values $q_1 = 0.49$ and $q_2 = 0.37$ \citep{cla13}, we find an ELC solution that gives $R_1 \simeq 7.9 \ R_\odot$, $R_2 \simeq 8.2 \ R_\odot$, $\omega \simeq 17.4 \ \rm{deg}$, and contamination as high as 5 \%. However, this solution has a higher $\chi^2$ than the models which allow triangularly sampled quadratic limb darkening coefficients \citep{kip13} to be free parameters, and it is important to consider that theoretical limb darkening values are poorly constrained for both giant stars and wide bandpasses. We therefore adopt the all-eclipse ELC solution in this work because it has the lowest $\chi^2$ and uses all available data to constrain the system.}
