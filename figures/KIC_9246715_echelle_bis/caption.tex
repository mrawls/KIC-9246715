\label{fig:echelle} \'Echelle diagram of the power density spectrum presented in Figure \ref{fig:seismo}. Darker regions correspond to larger peaks in power density. The power density spectrum is smoothed by a boxcar over seven bins and cut into 8.31-$\mu\rm{Hz}$ chunks; each is then stacked on top of its lower-frequency neighbor. This representation allows for visual identification of the modes. \revise{Lines are plotted to guide the eye toward a theoretical mode distribution according to the red giant universal pattern \citep{mos11}. It illustrates how we expect the modes to appear, but is not the result of a fit. Solid blue and dashed red lines are associated with the main (nominally Star 2) and marginally-detected (nominally Star 1) oscillations, respectively. The variable $\ell$ labels each mode by its spherical degree. Large spacing is $\Delta\nu = 8.31 \ \mu\rm{Hz}$ for the main (blue) lines and $8.60\ \mu\rm{Hz}$ for the secondary (red) lines (see Section \ref{subsubsec_second_osc}).}

  