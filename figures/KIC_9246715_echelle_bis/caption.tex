\label{fig:echelle} \'Echelle diagram of the power density spectrum presented in Figure \ref{fig:seismo}. Darker regions correspond to larger peaks in power density. The power density spectrum is smoothed by a boxcar over seven bins and cut into 8.31-$\mu\rm{Hz}$ chunks; each is then stacked on top of its lower-frequency neighbor. This representation allows for visual identification of the modes. \textbf{Solid blue and dashed red lines are plotted to guide the eye on mode distribution according to the red-giant universal pattern \citep{mos11}. It is an indication on how modes are distributed but not the result of a fitting. Solid blue and dashed red lines are associated with the main (nominally Star 2) and marginally-detected (nominally Star 1) oscillation set respectively. $\ell = 0$, $1$, $2$ indicate the mode degrees. Large spacing is $\Delta\nu = 8.31\ \mu\rm{Hz}$ for primary set and $8.60\ \mu\rm{Hz}$ for secondary set(see Section \ref{subsubsec_second_osc}).}

  