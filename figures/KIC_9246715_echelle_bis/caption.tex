\label{fig:echelle} \'Echelle diagram of the power density spectrum presented in Figure \ref{fig:seismo}. Darker regions correspond to larger peaks in power density. The power density spectrum is smoothed by a boxcar over nine bins and cut into 8.31-$\mu\rm{Hz}$ chunks; each is then stacked on top of its lower-frequency neighbor. This representation allows for visual identification of the modes. Solid blue lines (nominally Star 2) show expected mode distributions for angular degrees $\ell = 0$, $1$, $2$ with $\Delta\nu = 8.31\ \mu\rm{Hz}$, according to the red-giant universal pattern from \citet{mos11}. Dashed red lines (nominally Star 1) indicate the same for marginally detected modes with $\Delta\nu = 8.60\ \mu\rm{Hz}$ (see Section \ref{subsubsec_second_osc}).

  