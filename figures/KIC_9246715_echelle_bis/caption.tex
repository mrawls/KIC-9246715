\label{fig:echelle} \'Echelle diagram of the power density spectrum presented in Figure \ref{fig:seismo}. Darker regions correspond to larger peaks in power density. The power density spectrum is smoothed by a boxcar over nine bins then cut into 8.31-$\mu\rm{Hz}$ chunks, and each of them is then stacked on top of its lower-frequency neighbor. This representation allows for visual identification of the modes. Solid blue lines show expected mode distributions for angular (spherical) degree $\ell = 0$, $1$, and $2$, according to the red-giant universal pattern \citep{mos11} for $\Delta\nu = 8.31\ \mu\rm{Hz}$. Dashed red lines indicate the same for a star with $\Delta\nu = 8.60\ \mu\rm{Hz}$ (see Section \ref{subsubsec_second_osc}).

  