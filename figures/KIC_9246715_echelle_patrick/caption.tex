\label{fig:echelle} \'Echelle diagram of the power density spectrum presented in Figure \ref{fig:seismo}. Darker regions correspond to larger peaks in power density. The power density spectrum is cut into 8.31-$\mu$Hz chunks, and each of them is then stacked on top of its lower-frequency neighbor. This representation allows for visual identification of the modes. The left vertical alignment of peaks ($x$-axis between 0.5 and 4 $\mu$Hz) corresponds to angular (spherical) degree $l=1$, where we note the presence of mixed modes. The right alignment (4--8 $\mu$Hz) corresponds to $l=2$ on the left and $l=0$ on the right. ADD DIAGONAL LINES; THIS POSSIBLE SIGNATURE OF A SECOND OSCILLATING STAR IS WEAK AND DISCUSSED FURTHER IN SECTION \ref{search}.
  
  
  
  