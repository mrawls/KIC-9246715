\label{fig:echelle}  Echelle diagram of the power density spectrum presented in Fig. \ref{fig:seismo}. Black color corresponds to large peaks in power density. The power density spectrum was cut in 8.31-$\mu$Hz chunks, each of them was then stacked one on top of its lower-frequency neighbor. This representation allows for visible identification of the modes. The left vertical alignment of peaks ($x$-axis between 0.5 and 4 $\mu$Hz) corresponds to $l=1$ modes where we note the presence of mixed-modes. The right alignment (4-8 $\mu$Hz) corresponds to $l=2$ on the left part and $l=0$ on the right part. We do not distinguish any contamination of a possible companion oscillation spectrum. 
  
  
  
  