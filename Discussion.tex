\section{Discussion}\label{discuss}

\subsection{Asteroseismology}\label{seismo}
\begin{itemize}
\item Basically rehash and update the results presented in \citet{gau14}.
\item Emphasize that we only see oscillations from one star
\item Present asteroseismic mass, radius, evolutionary state, other things?
\item Include a figure with the star's power spectrum... JEAN I'm looking at you
\end{itemize}

\begin{equation}
\left( \frac{R}{R_\odot} \right) \simeq \left( \frac{\nu_{\rm{max}}}{\nu_{\rm{max,\odot}}} \right) \left( \frac{\Delta \nu}{\Delta \nu_\odot} \right)^{-2} \left( \frac{T_{\rm{eff}}}{T_{\rm{eff,\odot}}} \right)^{0.5}
\end{equation}

\begin{equation}
\left( \frac{M}{M_\odot} \right) \simeq \left( \frac{\nu_{\rm{max}}}{\nu_{\rm{max,\odot}}} \right)^3 \left( \frac{\Delta \nu}{\Delta \nu_\odot} \right)^{-4} \left( \frac{T_{\rm{eff}}}{T_{\rm{eff,\odot}}} \right)^{1.5}.
\end{equation}

\subsection{Stellar Activity}
\begin{itemize}
\item Discuss how different light curve ``chunks'' give different model results
\item Talk about any other weird or interesting things
\end{itemize}

\subsection{Results}\label{results}
\begin{itemize}
\item Quantitatively compare the Section \ref{model} results with the \citet{gau14} results described in Section \ref{seismo}
\item Make a clear figure that shows masses and radii from seismology vs. binary modeling
\end{itemize}

\subsection{Future Work}\label{future}
Model ALL the RG/EBs!
