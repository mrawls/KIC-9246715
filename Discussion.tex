\section{Discussion}\label{discuss}

\subsection{Comparison with asteroseismology}\label{seismo}

We expect both evolved giants in KIC 9246715 to exhibit solar-like oscillations. These should be observable as pure p-modes for radial oscillations ($\ell = 0$), mixed p- and g- modes for dipolar oscillations ($\ell = 1$), and p-dominated modes for quadrupolar oscillations ($\ell = 2$) in \emph{Kepler} long-cadence data. For solar-like oscillators, the average large frequency separation between consecutive p-modes of the same spherical degree $\ell$, $\Delta \nu$, has been shown to scale with the square root of the mean density of the star. The frequency of maximum oscillation power, $\nu_{\rm{max}}$, carries information about the physical conditions near the stellar surface and is a function of surface gravity and effective temperature \citep{kje95}. These scaling relations may be used to estimate a star's mean density and surface gravity:

\begin{equation} \label{density}
{\frac{\bar{\rho}}{\bar{\rho}_{\odot}}} \simeq {\left( \frac{\Delta \nu}{\Delta \nu_{\odot}} \right)}^{2}
\end{equation}

\noindent and

\begin{equation} \label{gravity}
{\frac{g}{g_{\odot}}} \simeq {\left( \frac{\nu_{\rm{max}}}{\nu_{\rm{max}, \ \odot}} \right)} {\left( \frac{T_{\rm{eff}}}{T_{\rm{eff}, \ \odot}} \right)}^{-1/2}.
\end{equation}

These relations should be valid only for oscillation modes of large radial order $n$, where pressure modes can be mathematically described in the frame of the ``asymptotic development'' \citep{tas80}. Even though red giants do not perfectly match these conditions, because the observed oscillation modes have radial orders $n < 10$, the scaling relations do appear to work. Quantifying how well they work and in what conditions is more challenging. This is why measuring oscillating stars' masses and radii independently from seismology is so important.

Surprisingly, when \citet{gau13} and \citet{gau14} analyzed the oscillation modes of KIC 9246715 to estimate global asteroseismic parameters, only one set of modes corresponding to a single oscillating star was found. Of the 17 oscillating RG/EBs in the \emph{Kepler} field, KIC 9246715 is the only one with a pair of giant stars (the rest are composed of a giant star and a main sequence star). The oscillation spectrum as well as its representation as an \'echelle diagram are shown in Figures \ref{fig:seismo} and \ref{fig:echelle}. Assuming a single oscillating star, the mode amplitudes are quite low ($A_{\rm{max}}(\ell=0) \simeq 14$ ppm, and not 6.6 ppm as erroneously reported by \citealt{gau14}) compared to the 20 ppm we expect based on mode amplitude scaling relations \citep{cor13}. In addition, the light curve displays photometric variability as large as 2\% peak-to-peak, which is typical of the signal created by spots on stellar surfaces. The pseudo-period of this variability was observed to be about half the orbital period, which suggests resonances in the system. \citet{gau14} speculated that star spots may be responsible for inhibiting oscillations on the smaller star, and a similar behavior was observed in other RG/EB systems. In this section, we reestimate the global seismic parameters of the oscillation spectrum that was previously identified (Section \ref{subsubsec_main_osc}), and we report the marginal detection of a second set of oscillations, where only a proxy of $\Delta\nu$ can be measured (Section \ref{subsubsec_second_osc}).

\subsubsection{Global asteroseismic parameters of the main oscillator}
\label{subsubsec_main_osc}
We now re-estimate $\nu_{\rm{max}}$ and $\Delta \nu$ for the oscillation spectrum in the same way as \citet{gau14}, but by using the whole \textit{Kepler} dataset (Q0--Q17). Differences with respect to previous estimates are negligible, as we find $\nu_{\rm{max}} = 106.4 \pm 0.8$ and $\Delta \nu = 8.31 \pm 0.01 \ \mu \rm{Hz}$. Because the ELC results yield $T_2/T_1=0.989$ (Table \ref{table1}) and the stellar atmosphere analysis gives $T_1 = 4990 \pm 90 \ \rm{K}$ and $T_2 = 5030 \pm 80 \ \rm{K}$ (Section \ref{parameters}), we assume an effective temperature $T_{\rm{eff}} = 5000 \pm 100 \ \rm{K}$ in the asteroseismic scaling equations. To determine mass, radius, surface gravity, and mean density, we use the scaling relations after correcting $\Delta \nu$ for the red giant regime \citep{mos13}. In essence, instead of directly plugging the observed $\Delta \nu_{\rm{obs}}$ into Equations \ref{density} and \ref{gravity}, we estimate the asymptotic large spacing via $\Delta \nu_{\rm{as}} = \Delta \nu_{\rm{obs}} (1 + \zeta)$, where $\zeta = 0.038$. With this correction of the large spacing, we obtain $M = 2.17 \pm 0.12 \ M_{\odot}$ and $R = 8.26 \pm 0.16 \ R_{\odot}$. In terms of mean density and surface gravity, which independently test the $\Delta \nu$  and $\nu_{\rm{max}}$ relations, respectively, we find $\bar{\rho}/\bar{\rho}_{\odot} = (3.862 \pm 0.009) \times 10 ^{-3}$ and $\log g = 2.942 \pm 0.007$. A comparison of key parameters determined from all our different modeling techniques is in Table \ref{table2}.

Based on the distribution of mixed modes, \citet{gau14} reported that the oscillation pattern period spacing was typical of that of a star from the secondary red clump, i.e., a core-He-burning star that has not experienced a helium flash. However, this conclusion was based on a period spacing of $\Delta \Pi \simeq 150 \ \rm{sec}$ (Mosser, private communication) which is difficult to measure robustly from a noisy oscillation spectrum. Red giant branch stars have smaller period spacings than red clump stars, and ($\Delta \Pi = 150 \ \rm{sec}$, $\Delta \nu = 8.31 \ \mu \rm{Hz}$) puts the oscillating star on the very edge of the asteroseismic parameter space that defines the secondary red clump \citep{mos14}. Therefore, while asteroseismology does indicate KIC 9246715 is a red clump star, there is a large uncertainty attached to the classification. This result is supported statistically by \citet{mig14}, who report it is more likely to find red clump stars than red giant branch stars in asteroseismic binaries in \emph{Kepler} data {\textbf{(WHY THAT IN FEW WORDS? ... working on that... )}}. We note that due to the large noise level of the mixed modes, we are unable to measure a core rotation rate.

\subsubsection{Identifying the oscillating star}
The asteroseismic mass and surface gravity are consistent with those from the ELC model for both stars, while the asteroseismic radius is only consistent with Star 2. Neither star's mean density agrees with the asteroseismic value, but Star 2 is much closer than Star 1. Overall, our asteroseismic analysis suggests the oscillating star is Star 2. However, we cannot definitely conclude this without considering the temperature dependence of the scaling relations. From \citet{gau13}, \citet{gau14}, and the present work, asteroseismic masses and radii were reported to be $(1.7 \pm 0.3 \ M_\odot, 7.7 \pm 0.4 \ R_\odot)$, $(2.06 \pm 0.13 \ M_\odot, 8.10 \pm 0.18 \ R_\odot)$, and $(2.17 \pm 0.12 \ M_\odot, 8.26 \pm 0.16 \ R_\odot)$, respectively. Among these, $\nu_{\rm{max}}$ does not vary much ($102.2, 106.4, 106.4 \ \mu\rm{Hz}$), and $\Delta \nu$ varies even less ($8.3, 8.32, 8.31 \ \mu\rm{Hz}$), while the assumed temperatures were 4699 K (from the KIC), 4857 K (from \citealt{hub14.2}), and 5000 K (this work).

Even if temperature is the least influential parameter on stellar masses and radii in the asteroseismic scalings, we are at a level of precision where errors on temperature dominate the global asteroseismic results. In general, asteroseismic scaling laws tend to better match mean density measurements than surface gravity measurements because of the latter's temperature dependence. While a more in-depth ``peak-bagging'' analysis of individual oscillation modes is beyond the scope of this paper, we strongly suspect the oscillating star is Star 2.

\subsubsection{Surface gravity disagreement}
The asteroseismic $\log g$ measurement nearly agrees with those from ELC, yet all three are some 0.3 dex lower than the spectroscopic $\log g$ values, as can be seen in Table \ref{table2}. This discrepancy is similar to the difference found for giant stars by \citet{hol15}. They investigate a large sample of stars from the ASPCAP (APOGEE Stellar Parameters and Chemical Abundances Pipeline) which have $\log g$ measured via spectroscopy and asteroseismology. They find that spectroscopic surface gravity measurements are roughly 0.2--0.3 dex too high for core-He-burning (red clump) stars and roughly 0.1--0.2 dex too high for shell-H-burning (red giant branch) stars. \citet{hol15} speculate the difference may be partially due to a lack of treatment of stellar rotation, and derive an empirical calibration relation for a ``correct'' $\log g$ for red giant branch stars only. However, the stars in KIC 9246715 do not rotate particularly fast ($v_{\rm{rot}} \sin i \simeq 8 \ \rm{km \ s}^{-1}$, as reported in Section \ref{parameters}), so we cannot dismiss this discrepancy so readily.

  