\section{Discussion}\label{discuss}

\subsection{Comparison with asteroseismology}\label{seismo}

We expect the evolved giants in KIC 9246715 to exhibit solar-like oscillations. These should be observable as p-modes for radial oscillations and mixed p- and g- modes for dipolar and quadrupolar oscillations in \emph{Kepler} long-cadence data. For solar-like oscillators, the average large frequency separation of p-modes $\Delta \nu$ has been shown to scale with the square root of the mean density of the star $\bar{\rho}$, while the frequency of maximum oscillation power $\nu_{\rm{max}}$ carries information about the physical conditions near the stellar surface and is a function of surface gravity $g$ and temperature $T_{\rm{eff}}$ \citep{kje95}. These scaling relations may be used to estimate a star's mean density and surface gravity:

\begin{equation} \label{density}
{\frac{\bar{\rho}}{\bar{\rho}_{\odot}}} \simeq {\left( \frac{\Delta \nu}{\Delta \nu_{\odot}} \right)}^{2}
\end{equation}

and

\begin{equation} \label{gravity}
{\frac{g}{g_{\odot}}} \simeq {\left( \frac{\nu_{\rm{max}}}{\nu_{\rm{max}, \ \odot}} \right)} {\left( \frac{T_{\rm{eff}}}{T_{\rm{eff}, \ \odot}} \right)}^{1/2}.
\end{equation}

These relations should be valid only for oscillation modes of large radial orders $n$, where pressure modes can be mathematically described in the frame of the ``asymptotic development'' \citep{tas80}. Even though red giants do not perfectly match these conditions, because the observed oscillation modes have radial orders $n < 10$, the scaling relations do appear to work (CITATION MAYBE??). Quantifying how well they work and in what conditions is more challenging. This is why measuring oscillating stars' masses and radii independently from seismology is so important.

When \citet{gau13} and \citet{gau14} analyzed the oscillation modes of KIC 9246715 to estimate global asteroseismic parameters, only one set of modes was found. Of the 15 oscillating red giants in eclipsing binaries in the \emph{Kepler} field, KIC 9246715 is the only one with a pair of giant stars (the rest are composed of a giant star and a main sequence star). The oscillation spectrum as well as its representation as an \'echelle diagram are shown in Figures \ref{fig:seismo} and \ref{fig:echelle}. \citealt{gau14} reported that the oscillation pattern was typical of that of a star from the secondary red clump, i.e., a star that burns He in the core, without having experienced an He flash. For a single oscillating star, the mode amplitudes are quite low ($A_{\rm{max}}(l=0) \simeq 14$ ppm, and not 6.6 as erroneously reported by \citealt{gau14}) with respect to the 20 ppm we expect based on mode amplitude scaling relations \citep{cor13}. In addition, the light curve displays photometric variability as large as 2\% peak-to-peak, which is typical of the signal created by spots on stellar surfaces. The pseudo-period of this variability was observed to be about half the orbital period, which suggests resonances in the system. \citet{gau14} speculated that star spots may be responsible for inhibiting oscillations on the smaller star, and a similar behavior was observed in five other RG/EB systems.

We now re-estimate $\nu_{\rm{max}}$ and $\Delta \nu$ for the oscillation spectrum in the same way as \citet{gau14}, but by using the whole \textit{Kepler} dataset (Q0--Q17). Differences with respect to previous estimates are negligible, as we find $\nu_{\rm{max}} = 106.4 \pm 0.8$ and $\Delta \nu = 8.31 \pm 0.01 \mu \rm{Hz}$. To determine mass, radius, surface gravity, and mean density, we use the scaling relations after correcting $\Delta \nu$ for the red giant regime \citep{mos13}. In essence, instead of directly plugging the observed $\Delta \nu_{\rm{obs}}$ into Equations \ref{density} and \ref{gravity}, we estimate the asymptotic large spacing $\Delta \nu_{\rm{as}}$ as follows: $\Delta \nu_{\rm{as}} = \Delta \nu_{\rm{obs}} (1 + \zeta)$, where $\zeta = 0.038$. With this correction of the large spacing, and assuming $T_{\rm{eff}} = 5050 \pm 100 \ \rm{K}$, we obtain $M = 2.21 \pm 0.12 \ M_{\odot}$ and $R = 8.30 \pm 0.16 \ R_{\odot}$. In terms of mean density and surface gravity, which independently test the $\Delta \nu$  and $\nu_{\rm{max}}$ relations, respectively, we find $\bar{\rho}/\bar{\rho}_{\odot} = (3.862 \pm 0.009)\ 10 ^{-3}$ and $\log g = 2.944 \pm 0.007$.

%\begin{equation} \label{radeq}
%\left( \frac{R}{R_\odot} \right) \simeq \left( \frac{\nu_{\rm{max}}}{\nu_{\rm{max, \ \odot}}} \right) \left( \frac{\Delta \nu}{\Delta \nu_\odot} \right)^{-2} {\left( \frac{T_{\rm{eff}}}{T_{\rm{eff, \ \odot}}} \right)}^{0.5}
%\end{equation}

%and

%\begin{equation} \label{masseq}
%{\left( \frac{M}{M_\odot} \right)} \simeq {\left( \frac{\nu_{\rm{max}}}{\nu_{\rm{max, \ \odot}}} \right)}^{3} {\left( \frac{\Delta \nu}{\Delta \nu_\odot} \right)}^{-4} {\left( \frac {T_{\rm{eff}}} {T_{\rm{eff, \ \odot}}} \right)}^{1.5}.
%\end{equation}

The asteroseismic mass and surface gravity are consistent with those from the ELC model for both stars, while the asteroseismic radius is only consistent with Star 2 (Table \ref{table2}). Neither star's mean density agrees with the asteroseismic value, but Star 2 is much closer than Star 1. Overall, our asteroseismic analysis suggests the oscillating star is Star 2. However, we cannot definitely conclude this without considering the temperature dependence of the scaling relations. From \citet{gau13}, \citet{gau14}, and the present work, asteroseismic masses and radii were reported to be $(1.7 \pm 0.3 M_\odot, 7.7 \pm 0.4 R_\odot)$ and $(2.06 \pm 0.13 M_\odot, 8.10 \pm 0.18 R_\odot)$, and now we have $(2.21 \pm 0.12 M_\odot, 8.30 \pm 0.16 R_\odot)$. Among these, $\nu_{\rm{max}}$ does not vary much (102.2, 106.4, and 106.4 $\mu\rm{Hz}$ respectively) while $\Delta \nu$ varies even less (8.3, 8.32, 8.31 $\mu$Hz), while the assumed temperatures were 4699 K (from the KIC), 4857 K (from \citet{hub14.2}), and 5050 K (present work JUSTIFY THIS TEMP CHOICE??). Even if temperature is the least influential parameter on stellar masses and radii in the asteroseismic scalings, we are at a level of precision where errors on temperature dominate the global asteroseismic results. This is also because of the temperature dependence that asteroseismic scalings better match mean density than surface gravity.

%We find that the masses, radii, and surface gravities for both stars in KIC 9246715 are all consistent with the asteroseismic values, so we unfortunately cannot use KIC 9246715 to draw conclusions about the accuracy of the scaling relations. A comparison of key parameters determined from our different modeling techniques is in Table \ref{table2}. We speculate on which star is the oscillator and further explore the binary's co-evolutionary history in Section \ref{context}.

%DISCREPANCY WITH MESA RADIUS RESULTS?? $\rightarrow$ TO BE DISCUSSED IN THE MESA SECTION
  
  
  
  
  
  
  
  