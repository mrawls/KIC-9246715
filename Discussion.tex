\section{Discussion and Results}\label{discuss}

\subsection{Asteroseismology}\label{seismo}
\begin{itemize}
\item Basicall rehash and update the results presented in \citet{gau14}.
\item Emphasize that we only see oscillations from one star
\item Present asteroseismic mass, radius, evolutionary state, other things?
\end{itemize}

\begin{equation}
\left( \frac{R}{R_\odot} \right) \simeq \left( \frac{\nu_{\rm{max}}}{\nu_{\rm{max,\odot}}} \right) \left( \frac{\Delta \nu}{\Delta \nu_\odot} \right)^{-2} \left( \frac{T_{\rm{eff}}}{T_{\rm{eff,\odot}}} \right)^{0.5}
\end{equation}

\begin{equation}
\left( \frac{M}{M_\odot} \right) \simeq \left( \frac{\nu_{\rm{max}}}{\nu_{\rm{max,\odot}}} \right)^3 \left( \frac{\Delta \nu}{\Delta \nu_\odot} \right)^{-4} \left( \frac{T_{\rm{eff}}}{T_{\rm{eff,\odot}}} \right)^{1.5}.
\end{equation}


% Fourth figure: Asteroseismology... probably some power spectra
%
%\begin{figure}[h!]
%\begin{center}
%\includegraphics[width=6 in]{fig4.eps}
%\end{center}
%\caption{Caption here.\label{seismofig}}
%\end{figure}

\subsection{Results}\label{results}
\begin{itemize}
\item Quantitatively compare the Section \ref{model} results with the \citet{gau14} results described in Section \ref{seismo}
\end{itemize}

% Fifth figure: Graphical representation of masses & radii, as a form of comparison
% between results from LC+RV and seismo. AKA, the money figure.
%
%\begin{figure}[h!]
%\begin{center}
%\includegraphics[width=6 in]{fig5.eps}
%\end{center}
%\caption{Caption here.\label{moneyfig}}
%\end{figure}
