\section{Discussion}\label{discuss}

\subsection{Comparison with asteroseismology}\label{seismo}

We expect the evolved giants in KIC 9246715 to exhibit solar-like oscillations. These should be observable as p-modes for radial oscillations and mixed p- and g- modes for dipolar and quadrupolar oscillations in \emph{Kepler} long-cadence data. For solar-like oscillators, the average large frequency separation of p-modes $\Delta \nu$ has been shown to scale with the square root of the mean density of the star $\bar{\rho}$, while the frequency of maximum oscillation power $\nu_{\rm{max}}$ carries information about the physical conditions near the stellar surface and is a function of surface gravity $g$ and temperature $T_{\rm{eff}}$ (CITE KJLEDSEN \& BEDDING 95). These scaling relations may be used to estimate a star's mean density and surface gravity:

\begin{equation} \label{density}
{\frac{\bar{\rho}}{\bar{\rho}_{\odot}}} \simeq {\left( \frac{\Delta \nu}{\Delta \nu_{\odot}} \right)}^{2}
\end{equation}

and

\begin{equation} \label{gravity}
{\frac{g}{g_{\odot}}} \simeq {\left( \frac{\nu_{\rm{max}}}{\nu_{\rm{max}, \ \odot}} \right)} {\left( \frac{T_{\rm{eff}}}{T_{\rm{eff}, \ \odot}} \right)}^{1/2}.
\end{equation}
These relations should be valid only for oscillation modes of large radial orders $n$, where pressure modes can be mathematically described in the frame of the  in the ``asymptotic development'' (CITE TASSOUL 1980). Even though red-giant do not match these conditions, because radial orders of observed modes are less than 10, the scaling relations look to work rather well, but we ignore until what extent. This is why measuring masses and radii, independently from seismology, of stars where we observe oscillation modes is so important.

When \citet{gau13} and \citet{gau14} analyzed the oscillation modes to estimate global asteroseismic parameters, only one set of modes was found. Of the 15 oscillating red giants in eclipsing binaries in the \emph{Kepler} field, KIC 9246715 is the only one with a pair of giant stars (the rest are composed of a giant star and a main sequence star). The oscillation spectrum as well as its representation as an \'echelle diagram is shown in Figure \ref{fig:seismo} (I'LL SEND THE PLOTS). \citealt{gau14} reported that the oscillation pattern was typical of that of a star from the secondary red clump, i.e. a star that burns He in the core, without having experienced an He flash. The mode amplitudes are quite low ($A_{\rm{max}}(l=0) \simeq 14$ ppm, and not 6.6 as erroneously reported by \citealt{gau14}) with respect to the 20 ppm we expect  based on the mode amplitude scaling relations (CORSARO ET AL 2013 MNRAS 430, 2313), and by supposing only one star oscillates. Besides, the light curve displays a significant photometric relative variability as large as 2\% peak-to-peak, which is typical of the signal created by spots on stellar surfaces. The pseudo period of this variability was observed to be about half the orbital period, which indicates resonances in the system. \citet{gau14} speculated that star spots may be responsible for inhibiting oscillations on the smaller star, as they observed on other five systems. 

We reestimated $\nu_{\rm{max}}$ and $\Delta\nu$ in the same way as \citet{gau14}, but by using the whole \textit{Kepler} dataset (Q0 to Q17). Differences with respect to previous estimates are minute but we keep the new ones as reference: $\nu_{\rm{max}} = 106.4 \pm 0.8$ and $\Delta\nu=8.31\pm0.01$. To determine mass, radius, surface gravity, and mean density, we use the scaling relations after correction of $\Delta\nu$ for the red-giant regime (CITE MOSSER 2013 A\&A 550, 126). In few words, instead of directly plugging the observed $\Delta\nu_{\rm{obs}}$ into the scaling equations, we estimate the asymptotic large spacing $\Delta\nu_{\rm{as}}$ as follows: $\Delta\nu_{\rm{as}} = \Delta\nu_{\rm{obs}} (1 + \zeta)$, where $\zeta = 0.038$. With this correction of the large spacing, and assuming $T_{\rm{eff}} = 5050 \pm 100$~K, we obtain: $M = 2.21 \pm 0.12 \ M_{\odot}$ and $R = 8.30 \pm 0.16 \ R_{\odot}$. In terms of mean density and surface gravity, which independently test $\Delta\nu$  and $\nu_{\rm{max}}$ relations respectively, we get:  $\bar{\rho}/\bar{\rho}_{\odot} = (3.862 \pm 0.009)\ 10 ^{-3}$, and $\log g = 2.944 \pm 0.007$ dex. 

%\begin{equation} \label{radeq}
%\left( \frac{R}{R_\odot} \right) \simeq \left( \frac{\nu_{\rm{max}}}{\nu_{\rm{max, \ \odot}}} \right) \left( \frac{\Delta \nu}{\Delta \nu_\odot} \right)^{-2} {\left( \frac{T_{\rm{eff}}}{T_{\rm{eff, \ \odot}}} \right)}^{0.5}
%\end{equation}

%and

%\begin{equation} \label{masseq}
%{\left( \frac{M}{M_\odot} \right)} \simeq {\left( \frac{\nu_{\rm{max}}}{\nu_{\rm{max, \ \odot}}} \right)}^{3} {\left( \frac{\Delta \nu}{\Delta \nu_\odot} \right)}^{-4} {\left( \frac {T_{\rm{eff}}} {T_{\rm{eff, \ \odot}}} \right)}^{1.5}.
%\end{equation}

The asteroseismic masses and surface gravity match those deducted from light curve and radial velocities measurements within the error bars, while radii match only star 2 (Table \ref{table2}). As regards mean densities, none of the stars match the asteroseismic estimates, but star 2 is much closer than star 1. Therefore, asteroseismic values tend to indicate that the star that displays oscillations is star 2. However, we cannot definitely conclude on that matter by considering more in details temperatures dependence. From the G13 and G14 papers, asteroseismic masses and radii were reported to be $(1.7\pm0.3 M_\odot, 7.7\pm0.4 R_\odot)$ and $(2.06\pm0.13 M_\odot, 8.10\pm0.18 R_\odot)$, and now we have $(2.21\pm0.12 M_\odot, 8.30\pm0.16 R_\odot)$. In these three papers (including the present), $\nu_{\rm{max}}$ does not vary much (102.2, 106.4, and 106.4 $\mu$Hz respectively) while $\Delta\nu$ varies even less (8.3, 8.32, 8.31 $\mu$Hz), while the assumed temperature were 4699, 4857, and 5050 K. Even if temperature is the least influent parameter on stellar masses and radii from scaling relations, we are at a level of precision were errors on temperature dominate on the asteroseismic output. This also because of the temperature dependence that asteroseismic scalings better match mean density than surface gravity.

%It is important to note the strong temperature dependence of these relations. WRITE SOMETHING ABOUT NEW M AND R ESTIMATES WITH BETTER TEMPERATURE INPUTS HERE. COMPARE LOG G VALUES HERE TOO SINCE IT'S JUST TESTING NU-MAX AND NOT DELTA-NU.

%We find that the masses, radii, and surface gravities for both stars in KIC 9246715 are all consistent with the asteroseismic values, so we unfortunately cannot use KIC 9246715 to draw conclusions about the accuracy of the scaling relations. A comparison of key parameters determined from our different modeling techniques is in Table \ref{table2}. We speculate on which star is the oscillator and further explore the binary's co-evolutionary history in Section \ref{context}.

%DISCREPANCY WITH MESA RADIUS RESULTS?? $\rightarrow$ TO BE DISCUSSED IN THE MESA SECTION

  
  
  
  
  
  
  
  
  
  
  
  
  
  
  
  
  
  
  
  
  
  
  
  
  
  
  
  
  
  
  
  
  
  
  
  
  
  