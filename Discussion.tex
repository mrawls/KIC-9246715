\section{Discussion}\label{discuss}

\subsection{Asteroseismology}\label{seismo}
This will briefly rehash the results from \citet{gau14} and point out that we don't do a detailed mode analysis, but we really do only see oscillations from one star, and these are the global parameters that result from asteroseismology (M, R, He burning red clump star, other things?). PATRICK, please write this section.

\begin{equation}
\left( \frac{R}{R_\s} \right) \simeq \left( \frac{\nu_{\rm{max}}}{\nu_{\rm{max,\s}}} \right) \left( \frac{\Delta \nu}{\Delta \nu_\s} \right)^{-2} \left( \frac{T_{\rm{eff}}}{T_{\rm{eff,\s}}} \right)^{0.5}
\end{equation}

\begin{equation}
\left( \frac{M}{M_\s} \right) \simeq \left( \frac{\nu_{\rm{max}}}{\nu_{\rm{max,\s}}} \right)^3 \left( \frac{\Delta \nu}{\Delta \nu_\s} \right)^{-4} \left( \frac{T_{\rm{eff}}}{T_{\rm{eff,\s}}} \right)^{1.5}.
\end{equation}


% PATRICK TIME

% Fourth figure: Asteroseismology... probably some power spectra
%
%\begin{figure}[h!]
%\begin{center}
%\includegraphics[width=6 in]{fig4.eps}
%\end{center}
%\caption{Caption here.\label{seismofig}}
%\end{figure}

\subsection{Results}\label{results}
Quantitatively compare the Section \ref{model} results with the \citet{gau14} results described in Section \ref{seismo}! Use at least one awesome figure.

% Fifth figure: Graphical representation of masses & radii, as a form of comparison
% between results from LC+RV and seismo. AKA, the money figure.
%
%\begin{figure}[h!]
%\begin{center}
%\includegraphics[width=6 in]{fig5.eps}
%\end{center}
%\caption{Caption here.\label{moneyfig}}
%\end{figure}
