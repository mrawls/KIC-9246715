\section{Discussion}\label{discuss}

\subsection{Comparison with asteroseismology}\label{seismo}

Because both stars in KIC 9246715 are evolved giants with convective envelopes, we expect both to exhibit solar-like oscillations. These should be observable as p-modes in \emph{Kepler} long-cadence data. For solar-like oscillators, the average large frequency separation of such modes $\Delta \nu$ has been shown to scale with the square root of the mean density of the star, while the frequency of maximum oscillation power $\nu_{\rm{max}}$ carries information about the physical conditions near the stellar surface and is a function of surface gravity and temperature \citep{cha13}. These scaling relations may be used to estimate a star's density and surface gravity:

\begin{equation} %\label{density}
{\left( {\frac{\rho}{\rho_{\odot}}} \right)} \simeq {\left( {\frac{\Delta \nu}{\Delta \nu_{\odot}}} \right)}^{2}
\end{equation}

and

\begin{equation} %\label{gravity}
{\left( \frac{g}{g_{\odot}} \right)} \simeq {\left( \frac{\Delta \nu_{\rm{max}}}{\Delta \nu_{\rm{max}, \ \odot}} \right)} {\left( \frac{T_{\rm{eff}}}{T_{\rm{eff}, \ \odot}} \right)}^{0.5}.
\end{equation}

However, when \citet{gau13} and \citet{gau14} analyzed the oscillation modes to estimate global asteroseismic parameters, only one set of modes was found. Of the 15 oscillating red giants in eclipsing binaries in the \emph{Kepler} field, KIC 9246715 is the only one with a pair of giant stars (the rest are composed of a giant star and a main sequence star). The oscillation spectrum is shown in Figure \ref{fig:seismo}. \citet{gau14} note that the mode amplitudes are low ($A_{\rm{max}} \simeq 6.6$ ppm), report photometric variability as large as 2\%, and speculate that star spots may be responsible for inhibiting oscillations on the smaller star. They also report $M = 2.06 \pm 0.13 \ M_{\odot}$ and $R = 8.10 \pm 0.18 \ R_{\odot}$ by assuming $T_{\rm{eff}} = 4857 \ \rm{K}$ and rearranging Equations \ref{density} and \ref{gravity} to yield

\begin{equation} %\label{radeq}
\left( \frac{R}{R_\odot} \right) \simeq \left( \frac{\nu_{\rm{max}}}{\nu_{\rm{max,\odot}}} \right) \left( \frac{\Delta \nu}{\Delta \nu_\odot} \right)^{-2} \left( \frac{T_{\rm{eff}}}{T_{\rm{eff,\odot}}} \right)^{0.5}
\end{equation}

and

\begin{equation} %\label{masseq}
\left( \frac{M}{M_\odot} \right) \simeq {\left( \frac{\nu_{\rm{max}}}{\nu_{\rm{max,\odot}}} \right)}^{3} \left( \frac{\Delta \nu}{\Delta \nu_\odot} \right)^{-4} \left( \frac {T_{\rm{eff}}} {T_{\rm{eff,\odot}}} \right)^{1.5}.
\end{equation}

It is important to note the strong temperature dependence of these relations. WRITE SOMETHING ABOUT NEW M AND R ESTIMATES WITH BETTER TEMPERATURE INPUTS HERE. COMPARE LOG G VALUES HERE TOO SINCE IT'S JUST TESTING NU-MAX AND NOT DELTA-NU.

We find that the masses, radii, and surface gravities for both stars in KIC 9246715 are all consistent with the asteroseismic values, so we unfortunately cannot use KIC 9246715 to draw conclusions about the accuracy of the scaling relations. We speculate on which star is the oscillator and further explore the binary's co-evolutionary history in Section \ref{context}.

We can say that the oscillating star, whichever it may be, is a core-helium-burning star on the red clump. WRITE ABOUT HOW WE KNOW THIS BECAUSE MIXED MODES.

\subsection{Searching for a second set of oscillations}

Given that the giants in KIC 9246715 are nearly twins, with $L_1/L_2 = 0.94$, $R_1/R_2 = 0.95$, $M_1/M_2 = 1.01$, and $T_1/T_2 = 1.01$, we test whether it is possible that we see only one set of oscillation modes because both stars are oscillating with virtually identical frequencies. It is unlikely that two sets of solar-like oscillations lie on top of one another, because the predicted difference in $\nu_{\rm{max}}$ for these not-quite-identical stars is about $11 \ \mu \rm{Hz}$ (from an inversion of Equations \ref{radeq} and \ref{masseq}), and the predicted difference in $\Delta \nu$ is about $0.7 \ \mu \rm{Hz}$. The resolution of the power spectrum in this region is VALUE HERE. Given this, the oscillation pattern from a second star (if present) should not appear to lie exactly on top of the oscillation pattern we do see.

We also investigate whether the modes show any frequency modulation as a function of orbital phase by examining portions of the power spectrum spanning less than the orbital period. However, the solar-like oscillations are both weak and short-lived, so it is difficult to clearly resolve any modes in a power spectrum of a light curve segment less than several hundred days long. At $\nu_{\rm{max}} = 106 \ \mu \rm{Hz}$, the maximum frequency shift expected from a $\pm \ 30 \ \rm{km} \ \rm{s}^{-1}$ radial velocity is $0.01 \ \mu \rm{Hz}$. WHICH IS LESS THAN THE RESOLUTION? We conclude that one star truly lacks visible oscillations while the other has low-amplitude oscillations, and we are unable to definitively say which star is which.
    
    