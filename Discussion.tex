\section{Discussion}\label{discuss}

\subsection{Comparison with asteroseismology}\label{seismo}

We expect the evolved giants in KIC 9246715 to exhibit solar-like oscillations. These should be observable as p-modes for radial oscillations and mixed p- and g- modes for dipolar and quadrupolar oscillations in \emph{Kepler} long-cadence data. For solar-like oscillators, the average large frequency separation of p-modes $\Delta \nu$ has been shown to scale with the square root of the mean density of the star, while the frequency of maximum oscillation power $\nu_{\rm{max}}$ carries information about the physical conditions near the stellar surface and is a function of surface gravity and temperature \citep{cha13} (CITE KJLEDSEN \& BEDDING 95 instead). These scaling relations may be used to estimate a star's density and surface gravity:

\begin{equation} \label{density}
{\frac{\bar{\rho}}{\bar{\rho}_{\odot}}} \simeq {\left( \frac{\Delta \nu}{\Delta \nu_{\odot}} \right)}^{2}
\end{equation}

and

\begin{equation} \label{gravity}
{\frac{g}{g_{\odot}}} \simeq {\left( \frac{\nu_{\rm{max}}}{\nu_{\rm{max}, \ \odot}} \right)} {\left( \frac{T_{\rm{eff}}}{T_{\rm{eff}, \ \odot}} \right)}^{1/2}.
\end{equation}

However, when \citet{gau13} and \citet{gau14} analyzed the oscillation modes to estimate global asteroseismic parameters, only one set of modes was found. Of the 15 oscillating red giants in eclipsing binaries in the \emph{Kepler} field, KIC 9246715 is the only one with a pair of giant stars (the rest are composed of a giant star and a main sequence star). The oscillation spectrum as well as its representation as an \'echelle diagram is shown in Figure \ref{fig:seismo} (I'LL SEND THE PLOTS). The mode amplitudes are quite low ($A_{\rm{max}}(l=0) \simeq 14$ ppm, and not 6.6 as erroneously reported by \citealt{gau14}) with respect to the 20 ppm we expect by supposing only one star oscillates, based on the mode amplitude scaling relations (CORSARO ET AL 2013). Besides, the light curve displays a significant photometric relative variability as large as 2\% peak-to-peak. \citet{gau14} speculated that star spots may be responsible for inhibiting oscillations on the smaller star, as they observed on other five systems. Even though the star's oscillation was analyzed by \citet{gau14}, we reestimated $\nu_{\rm{max}}$ and $\Delta\nu$ in the same way, but by using the whole \textit{Kepler} dataset (Q0 to Q17). Differences with respect to previous estimates are minute but we keep the new ones as reference: $\nu_{\rm{max}} = 106.4 \pm 0.8$ and $\Delta\nu=8.31\pm0.01$. To determine mass, radius, surface gravity, and mean density, we use the scaling relations after correction of $\Delta\nu$ for the red-giant regime (CITE MOSSER 2012). In few words, instead of directly plugging the observed $\Delta\nu_{\rm{obs}}$ into the scaling equations, we estimate the asymptotic large spacing $\Delta\nu_{\rm{as}}$ as follows: $\Delta\nu_{\rm{as}} = \Delta\nu_{\rm{obs}} (1 + \zeta)$, where $\zeta = 0.038$.
They also report $M = 2.06 \pm 0.13 \ M_{\odot}$ and $R = 8.10 \pm 0.18 \ R_{\odot}$ by assuming $T_{\rm{eff}} = 4857 \ \rm{K}$ and rearranging Equations \ref{density} and \ref{gravity} to yield

%\begin{equation} \label{radeq}
%\left( \frac{R}{R_\odot} \right) \simeq \left( \frac{\nu_{\rm{max}}}{\nu_{\rm{max, \ \odot}}} \right) \left( \frac{\Delta \nu}{\Delta \nu_\odot} \right)^{-2} {\left( \frac{T_{\rm{eff}}}{T_{\rm{eff, \ \odot}}} \right)}^{0.5}
%\end{equation}

%and

%\begin{equation} \label{masseq}
%{\left( \frac{M}{M_\odot} \right)} \simeq {\left( \frac{\nu_{\rm{max}}}{\nu_{\rm{max, \ \odot}}} \right)}^{3} {\left( \frac{\Delta \nu}{\Delta \nu_\odot} \right)}^{-4} {\left( \frac {T_{\rm{eff}}} {T_{\rm{eff, \ \odot}}} \right)}^{1.5}.
%\end{equation}

It is important to note the strong temperature dependence of these relations. WRITE SOMETHING ABOUT NEW M AND R ESTIMATES WITH BETTER TEMPERATURE INPUTS HERE. COMPARE LOG G VALUES HERE TOO SINCE IT'S JUST TESTING NU-MAX AND NOT DELTA-NU.

We find that the masses, radii, and surface gravities for both stars in KIC 9246715 are all consistent with the asteroseismic values, so we unfortunately cannot use KIC 9246715 to draw conclusions about the accuracy of the scaling relations. A comparison of key parameters determined from our different modeling techniques is in Table \ref{table2}. We speculate on which star is the oscillator and further explore the binary's co-evolutionary history in Section \ref{context}.

We can say that the oscillating star, whichever it may be, is a core-helium-burning star on the red clump. WRITE ABOUT HOW WE KNOW THIS BECAUSE MIXED MODES. DISCREPANCY WITH MESA RADIUS RESULTS??

  
  
  
  
  
  
  
  
  
  