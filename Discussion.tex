\section{Discussion}\label{discuss}

\subsection{Comparison with asteroseismology}\label{seismo}
We expect both evolved giants in KIC 9246715 to exhibit solar-like oscillations. These should be observable as pure p-modes for radial oscillations ($\ell = 0$), mixed p- and g- modes for dipolar oscillations ($\ell = 1$), and p-dominated modes for quadrupolar oscillations ($\ell = 2$) in \emph{Kepler} long-cadence data. For solar-like oscillators, the average large frequency separation between consecutive p-modes of the same spherical degree $\ell$, $\Delta \nu$, has been shown to scale with the square root of the mean density of the star. The frequency of maximum oscillation power, $\nu_{\rm{max}}$, carries information about the physical conditions near the stellar surface and is a function of surface gravity and effective temperature \citep{kje95}. These scaling relations may be used to estimate a star's mean density and surface gravity:

\begin{equation} \label{density}
{\frac{\bar{\rho}}{\bar{\rho}_{\odot}}} \simeq {\left( \frac{\Delta \nu}{\Delta \nu_{\odot}} \right)}^{2}
\end{equation}

\noindent and

\begin{equation} \label{gravity}
{\frac{g}{g_{\odot}}} \simeq {\left( \frac{\nu_{\rm{max}}}{\nu_{\rm{max}, \ \odot}} \right)} {\left( \frac{T_{\rm{eff}}}{T_{\rm{eff}, \ \odot}} \right)}^{-1/2}.
\end{equation}

\revise{Equation} \ref{density} is valid only for oscillation modes of large radial order $n$, where pressure modes can be mathematically described in the frame of the asymptotic development \citep{tas80}. Even though red giants do not perfectly match these conditions, because the observed oscillation modes have \revise{small radial orders on the order of $n \sim 10$}, the scaling relations do appear to work. Quantifying how well they work and in what conditions is more challenging. This is why measuring oscillating stars' masses and radii independently from seismology is so important.

Surprisingly, when \citet{gau13} and \citet{gau14} analyzed the oscillation modes of KIC 9246715 to estimate global asteroseismic parameters, only one set of modes corresponding to a single oscillating star was found. Of the 18 oscillating RG/EBs in the \emph{Kepler} field, KIC 9246715 is the only one with a pair of giant stars (the rest are composed of a giant star and a main sequence star).
%The oscillation spectrum as well as its representation as an \'echelle diagram are shown in Figures \ref{fig:seismo} and \ref{fig:echelle} \revise{(see Section \ref{subsec_second_osc} for a discussion of the marginal detection of a second set of oscillations illustrated in Figure \ref{fig:echelle})}.
Assuming a single oscillating star, the mode amplitudes are quite low ($A_{\rm{max}}(\ell=0) \simeq 14$ ppm, and not 6.6 ppm as erroneously reported by \citealt{gau14}) compared to the 20 ppm we expect based on mode amplitude scaling relations \citep{cor13}. In addition, the light curve displays photometric variability as large as 2\% peak-to-peak, \revise{as shown in Figure \ref{fig:lcfig2}}, which is typical of the signal created by spots on stellar surfaces. The pseudo-period of this variability was observed to be about half the orbital period, which suggests resonances in the system. \citet{gau14} speculated that star spots may be responsible for inhibiting oscillations on the smaller star, and a similar behavior was observed in other RG/EB systems. In this section, we reestimate the global seismic parameters of the oscillation spectrum that was previously identified (Section \ref{subsubsec_main_osc}), \revise{analyze the mixed oscillation modes to determine the oscillating star's evolutionary state (Section \ref{subsubsec_mixed}), identify which star is more likely to be exhibiting oscillations (Section \ref{identifying}), and address the discrepancy between different surface gravity measurements (Section \ref{gravity_compare}).}

\subsubsection{Global asteroseismic parameters of the oscillating star}
\label{subsubsec_main_osc}
We now re-estimate $\nu_{\rm{max}}$ and $\Delta \nu$ for the oscillation spectrum in the same way as \citet{gau14}, but by using the whole \textit{Kepler} dataset (Q0--Q17). \revise{The frequency at maximum amplitude of solar-like oscillations $\nu_{\rm{max}}$ is measured by fitting the mode envelope with a Gaussian function and the background stellar activity with a sum of two semi-Lorentzians. The large frequency separation $\Delta\nu$ is obtained from the filtered autocorrelation of the time series \citep{mos09}.} Differences with respect to previous estimates are negligible, as we find $\nu_{\rm{max}} = 106.4 \pm 0.8$ and $\Delta \nu = 8.31 \pm 0.02 \ \mu \rm{Hz}$. Because the ELC results yield $T_2/T_1=0.989$ (Table \ref{table1}) and the stellar atmosphere analysis gives $T_1 = 4990 \pm 90 \ \rm{K}$ and $T_2 = 5030 \pm 80 \ \rm{K}$ (Section \ref{parameters}), we assume an effective temperature $T_{\rm{eff}} = 5000 \pm 100 \ \rm{K}$ in the asteroseismic scaling equations. To determine mass, radius, surface gravity, and mean density, we use the scaling relations after correcting $\Delta \nu$ for the red giant regime \citep{mos13}\footnote{\newrevise{Other scaling relation applications, such as \citet{cha11} or \citet{kal10}, assume the observed $\Delta \nu$ is equal to the asymptotic $\Delta \nu$. \citet{mos13} uses a correction factor to account for the fact that oscillating red giants are not in the asymptotic regime, which we apply here}}. In essence, instead of directly plugging the observed $\Delta \nu_{\rm{obs}}$ into Equations \ref{density} and \ref{gravity}, we estimate the asymptotic large spacing via $\Delta \nu_{\rm{as}} = \Delta \nu_{\rm{obs}} (1 + \zeta)$, where $\zeta = 0.038$. With this correction of the large spacing, we obtain $M = 2.17 \pm 0.14 \ M_{\odot}$ and $R = 8.26 \pm 0.18 \ R_{\odot}$. In terms of mean density and surface gravity, which independently test the $\Delta \nu$  and $\nu_{\rm{max}}$ relations, respectively, we find $\bar{\rho}/\bar{\rho}_{\odot} = (3.86 \pm 0.02) \times 10 ^{-3}$ and $\log g = 2.942 \pm 0.008$. A comparison of key parameters determined from all our different modeling techniques is in Table \ref{table2}.


  