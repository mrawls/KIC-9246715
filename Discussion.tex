\section{Discussion}\label{discuss}

\subsection{Comparison with asteroseismology}\label{seismo}

Because both stars in KIC 9246715 are evolved giants with convective envelopes, we expect both to exhibit solar-like oscillations. These should be observable as p-modes in \emph{Kepler} long-cadence data. However, when \citet{gau13} and \citet{gau14} analyzed the oscillation modes to estimate global asteroseismic parameters, only one set of modes was found. KIC 9246715 is one of 15 oscillating red giants in eclipsing binaries in the \emph{Kepler} field, but the only one with a pair of giant stars (the rest are composed of a giant star and a main sequence star). The oscillation spectrum is shown in Figure \ref{fig:seismo}. \citet{gau14} note that the mode amplitudes are low ($A_{\rm{max}} \simeq 6.6$ ppm), report photometric variability as large as 2\%, and speculate that star spots may be responsible for inhibiting oscillations on the smaller star. They also report $M = 2.06 \pm 0.13 \ M_{\odot}$ and $R = 8.10 \pm 0.18 \ R_{\odot}$ using the asteroseismic scaling relations

\begin{equation}
\left( \frac{R}{R_\odot} \right) \simeq \left( \frac{\nu_{\rm{max}}}{\nu_{\rm{max,\odot}}} \right) \left( \frac{\Delta \nu}{\Delta \nu_\odot} \right)^{-2} \left( \frac{T_{\rm{eff}}}{T_{\rm{eff,\odot}}} \right)^{0.5}
\end{equation}

and

\begin{equation}
\left( \frac{M}{M_\odot} \right) \simeq {\left( \frac{\nu_{\rm{max}}}{\nu_{\rm{max,\odot}}} \right)}^{3} \left( \frac{\Delta \nu}{\Delta \nu_\odot} \right)^{-4} \left( \frac{T_{\rm{eff}}}{T_{\rm{eff,\odot}}} \right)^{1.5}.
\end{equation}

It is important to note the strong temperature dependence of these relations. WRITE SOMETHING ABOUT NEW M AND R ESTIMATES WITH BETTER TEMPERATURE INPUTS HERE. 

WRITE ABOUT HOW WE KNOW THE OSCILLATING STAR IS ON THE RED CLUMP BECAUSE MIXED MODES HERE.

Given that the giants in KIC 9246715 are very nearly twins, with $L_1/L_2 = 0.94$, $R_1/R_2 = 0.95$, $M_1/M_2 = 1.01$, and $T_1/T_2 = 1.01$, we test whether it is possible that we see only one set of oscillation modes because both stars are oscillating with virtually identical frequencies. 

WRITE ABOUT WHY THAT DOESN'T QUITE WORK BECAUSE OUR RESOLUTION IS BETTER THAN THE PREDICTED DIFFERENCE IN MODE FREQUENCIES. ALSO WRITE ABOUT HOW WE LOOK FOR FREQUENCY MODULATIONS AS A FUNCTION OF WHERE THE BINARY IS IN ITS ORBIT TO SEARCH FOR A SECOND SET OF MODES, BUT THIS IS HARD/INCONCLUSIVE.

We conclude that one star truly lacks visible oscillations while the other has low-amplitude oscillations, and we are unable to definitively say which star is which. (right??)