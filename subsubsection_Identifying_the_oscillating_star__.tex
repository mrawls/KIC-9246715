\subsubsection{Identifying the oscillating star}\label{identifying}
The asteroseismic mass and \revise{radius are consistent with those from the ELC model for both stars. The surface gravity of Star 2 nearly agrees with the asteroseismic value, while the gravity of Star 1 does not.} Neither star's mean density agrees with the asteroseismic value, but Star 2 is closer than Star 1. Overall, this suggests the source of the oscillations is Star 2. However, we cannot definitely conclude this without considering the temperature dependence of the scaling relations. From \citet{gau13}, \citet{gau14}, and the present work, asteroseismic masses and radii were reported to be $(1.7 \pm 0.3 \ M_\odot, 7.7 \pm 0.4 \ R_\odot)$, $(2.06 \pm 0.13 \ M_\odot, 8.10 \pm 0.18 \ R_\odot)$, and $(2.17 \pm 0.12 \ M_\odot, 8.26 \pm 0.16 \ R_\odot)$, respectively. Among these, $\nu_{\rm{max}}$ does not vary much ($102.2, 106.4, 106.4 \ \mu\rm{Hz}$), and $\Delta \nu$ varies even less ($8.3, 8.32, 8.31 \ \mu\rm{Hz}$), while the assumed temperatures were 4699 K (from the KIC), 4857 K (from \citealt{hub14.2}), and 5000 K (this work).

\revise{Since the scaling equations can be combined to give a mean density that is independent of temperature and $\nu_{\rm max}$ (Equation \ref{density}), one might na\.ively expect a better asteroseismic estimation of density compared to surface gravity.} Even if temperature is the least influential parameter on stellar masses and radii in the asteroseismic scalings, we are at a level of precision where errors on temperature dominate the global asteroseismic results. While a more in-depth ``peak-bagging'' analysis of individual oscillation modes is beyond the scope of this paper, we strongly suspect the oscillating star is Star 2.

\subsubsection{Surface gravity disagreement}\label{gravity_compare}
The asteroseismic $\log g$ measurement nearly agrees with those from ELC, yet all three are some 0.3 dex lower than the spectroscopic $\log g$ values, as can be seen in Table \ref{table2}. This discrepancy is similar to the difference found for giant stars by \citet{hol15}. They investigate a large sample of stars from the ASPCAP (APOGEE Stellar Parameters and Chemical Abundances Pipeline) which have $\log g$ measured via spectroscopy and asteroseismology. They find that spectroscopic surface gravity measurements are roughly 0.2--0.3 dex too high for core-He-burning (red clump) stars and roughly 0.1--0.2 dex too high for shell-H-burning (red giant branch) stars. \citet{hol15} speculate the difference may be partially due to a lack of treatment of stellar rotation, and derive an empirical calibration relation for a ``correct'' $\log g$ for red giant branch stars only. However, the stars in KIC 9246715 do not rotate particularly fast ($v_{\rm{rot}} \sin i \lesssim 8 \ \rm{km \ s}^{-1}$, which includes a contribution from macroturbulence as discussed in Section \ref{parameters}), so we cannot dismiss this discrepancy so readily.