\subsection{Signatures of stellar activity}\label{actrot}
KIC 9246715 is an interesting pair of well-separated red giants that exhibit photometric variations from stellar activity, weak or absent solar-like oscillations, and a notably eccentric orbit. In this and the following section, we discuss how stellar activity and tidal forces have acted over the binary's lifetime to arrive at the system we see today. \revise{The first confirmed case of activity and/or tides suppressing convection-driven oscillations was \citet{der11}, and as \citet{gau14} showed, they likely play an important role in many RG/EBs.}

\revise{In this system, the light curve residuals discussed in Section \ref{segment} and Figure \ref{fig:ELCresult} show significant scatter during both eclipses, and especially primary eclipse (when Star 1 is in front).} This means at least Star 1 is magnetically active, and activity in the system is further supported by photometric variability of up to 2\% on a timescale approximately equal to half the orbital period \citep{gau14}. A magnetically active Star 1 is also consistent with Star 2 as the suspected main oscillator, because strong magnetic fields may be responsible for damping solar-like oscillations.

Figures \ref{fig:emission1} and \ref{fig:emission2} investigate whether magnetic activity has any appreciable effect on absorption lines in either star. Following the approach of \citet{fro12}, we plot each target spectrum (solid colored line) on top of a model (dotted line), and show the difference below (solid black line). The model spectrum is a PHOENIX BT-Settl stellar atmosphere like the one described in Section \ref{bf} \citep{all03,asp09}, with $T_{\rm{eff}} = 5000$ and $\log g = 3.0$. It has been convolved to a lower resolution much closer to that of the ARCES and TRES spectrographs.

We examine a selection of the strongest {\rm Fe}\kern 0.1em{\sc i} lines which fall in the disentangled wavelength region and are either prone to Zeeman splitting in the presence of strong magnetic fields \citep{har73}, or not \citep{sis70}. The non-magnetic lines serve as a control. We find none of the six panels of {\rm Fe}\kern 0.1em{\sc i} absorption lines in either star show any significant deviation from the model spectrum. Thus, there is no apparent Zeeman broadening, which is unsurprising for evolved red giants. Magnetic fields must be quite strong to produce this effect. However, the $\rm{H}\alpha$ and {\rm Ca}\kern 0.1em{\sc ii} absorption lines, which can be indicators of chromospheric activity, are somewhat more interesting. The $\rm{H}\alpha$ line appears significantly deeper and broader than the model in both stars. While net emission is typically associated with activity, \citet{rob90} show several examples of main sequence stars with increased $\rm{H}\alpha$ absorption due to chromospheric heating, although they caution it is difficult to separate the photospheric and chromospheric contributions to the line. Still, the increased $\rm{H}\alpha$ absorption equivalent width is slightly more pronounced in Star 1 than Star 2. \revise{While this may not be a significant difference on its own, taken together with the stellar activity from Section \ref{segment}, it also suggests Star 1 is the more magnetically active of the pair.} It is unclear whether the {\rm Ca}\kern 0.1em{\sc ii} doublet shows signs of excess broadening or increased equivalent width, but these lines certainly do not have \emph{smaller} equivalent widths than the model.

\revise{The overall photometric variability from Figure \ref{fig:lcfig2} and increased scatter in the primary eclipse residual from Figure \ref{fig:ELCfig} indicate that both stars are moderately magnetically active, and Star 1 more so than Star 2. This is consistent with increased $\rm{H}\alpha$ absorption in both stars (and especially Star 1),} and supports our suspicion that Star 2 is the main oscillator, and that stellar activity is suppressing solar-like oscillations in Star 1.
