\subsection{Stellar activity and rotation}\label{actrot}
The double red giant eclipsing binary KIC 9246715 is an interesting system with photometric variations from stellar activity and a very eccentric orbit given its age, long orbital period, and well-separated nature. In this and the following section, we discuss how stellar activity and tidal forces have likely interacted over the binary's lifetime to arrive at the system we see today.

Figures \ref{fig:emission1} and \ref{fig:emission2} investigate whether magnetic activity has any appreciable effect on absorption lines in either star. Following the approach of \citet{fro12}, we plot each target spectrum (colored line) on top of a model (dotted line), and show the difference below (solid black line). The model spectrum is similar to the one described in Section \ref{bf} \citep[PHOENIX BT-Settl, ][]{all03}, but now has a more representative $T_{\rm{eff}} = 5000$ and $\log g = 3.0$. It has also been convolved to a lower resolution much closer to that of the ARCES and TRES spectrographs.

First, we examine a selection of the strongest {\rm Fe}\kern 0.1em{\sc i} lines which fall in the disentangled wavelength region and are either prone to Zeeman splitting in the presence of strong magnetic fields \citep{har73}, or not \citep{sis70}. 

Discuss H alpha and Ca II.

Discuss vsini things.
  
  
  