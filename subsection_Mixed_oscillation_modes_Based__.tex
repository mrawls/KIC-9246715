\subsection{Mixed oscillation modes}
Based on the distribution of mixed modes, \citet{gau14} reported that the oscillation pattern period spacing was typical of that of a star from the secondary red clump, i.e., a core-He-burning star that has not experienced a helium flash. However, this conclusion is based on a period spacing of $\Delta \Pi \simeq 150 \ \rm{sec}$, which is difficult to measure robustly from a noisy oscillation spectrum. Red giant branch stars have smaller period spacings than red clump stars, and ($\Delta \Pi = 150 \ \rm{sec}$, $\Delta \nu = 8.31 \ \mu \rm{Hz}$) puts the oscillating star on the very edge of the asteroseismic parameter space that defines the secondary red clump \citep{mos14}. Therefore, while asteroseismology does indicate the oscillator in KIC 9246715 is a red clump star, there is a large uncertainty attached to the classification. This result is supported statistically by \citet{mig14}, who report it is more likely to find red clump stars than red giant branch stars in asteroseismic binaries in \emph{Kepler} data. This is largely due to the fact that evolved stars spend more time on the horizontal branch than the red giant branch. We note that due to the large noise level of the mixed modes, we are unable to measure a core rotation rate in the manner of \citet{bec12} and \citet{mos12}.
\revise{MEREDITH MOVED THE PARAGRAPH ABOVE FROM SECTION \ref{identifying} AND OMITTED THE CITATION TO (Mosser, Private Communication).}