%\begin{abstract}
We combine \emph{Kepler} photometry with ground-based radial velocities to present a comprehensive photodynamical model for the double red giant eclipsing binary KIC 9246715. The two stars are nearly identical in mass ($M_1 = 2.16 \pm 0.04\ M_{\odot}$, $M_2 = 2.14 \pm 0.03\ M_{\odot}$), radius ($R_1 = 7.90 \pm 0.04 \ R_{\odot}$, $R_2 = 8.33 \pm 0.04 \ R_{\odot}$), and surface gravity (PUT SURFACE GRAVITIES HERE) in a well-separated and eccentric ($e = 0.35$) 171-day orbit, yet an asteroseismic analysis identifies only a single set of solar-like oscillations. The oscillation amplitudes are weaker than expected from similar red giants. Because the two stars are nearly twins, KIC 9246715 is not a good target for precisely testing the asteroseismic scaling relations---both stars are consistent with the single inferred mass, radius, and surface gravity ($M_{\rm{seismo}} = 2.06 \pm 0.13 \ M_{\odot}$, $R_{\rm{seismo}} = 8.10 \pm 0.18 \ R_{\odot}$, PUT SEISMIC LOG G HERE). We suspect stellar activity or modest tidal forces are responsible for a lack of solar-like oscillations in one star and weak oscillations in the other. We further show that KIC 9246715 is a coeval binary with both stars in the Helium-core-burning red clump. This system is a useful asteroseismic case study and paves the way for a detailed analysis of more red giants in eclipsing binaries.
%\end{abstract}
    
    