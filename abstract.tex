%\begin{abstract}
{\textbf{Mention the marginal detection of the second set of oscillations.}} We combine \emph{Kepler} photometry with ground-based spectra to present a comprehensive dynamical model of the double red giant eclipsing binary KIC 9246715. While the two stars are very similar in mass ($M_1 = 2.16 \pm 0.04\ M_{\odot}$, $M_2 = 2.14 \pm 0.03\ M_{\odot}$) and radius ($R_1 = 7.90 \pm 0.04 \ R_{\odot}$, $R_2 = 8.33 \pm 0.04 \ R_{\odot}$), an asteroseismic analysis identifies only a single set of solar-like oscillations with unusually low amplitude. Because the two stars are nearly twins, KIC 9246715 is a difficult target for a precise test of the asteroseismic scaling relations, which yield $M = 2.17 \pm 0.13 \ M_{\odot}$ and $R = 8.26 \pm 0.16 \ R_{\odot}$. Both Star 1 and Star 2 are consistent with the single inferred asteroseismic mass, but the asteroseismic radius agrees with Star 2's radius only. We therefore suspect the oscillating star is Star 2, the slightly larger of the pair. We find evidence for stellar activity and modest tidal forces acting over the 171-day eccentric orbit, which are likely responsible for the lack of solar-like oscillations in one star and weak oscillations in the other. The two stars are coeval, but stellar evolution modeling suggests they are on the red giant branch (shell-H-burning) while asteroseismology suggests they are on the secondary red clump (core-He-burning). This system is a useful case study and paves the way for a detailed analysis of more red giants in eclipsing binaries, the best benchmark we have for asteroseismology.
%\end{abstract}

  
  
  
  