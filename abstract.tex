%\begin{abstract}
We combine \emph{Kepler} photometry with ground-based spectra to present a comprehensive dynamical model of the double red giant eclipsing binary KIC 9246715. While the two stars are very similar in mass ($M_1 = 2.171\substack{+0.006 \\ -0.008} \ M_{\odot}$, $M_2 = 2.149\substack{+0.006 \\ -0.008} \ M_{\odot}$) and radius ($R_1 = 8.37\substack{+0.03 \\ -0.07} \ R_{\odot}$, $R_2 = 8.30\substack{+0.04 \\ -0.03} \ R_{\odot}$), an asteroseismic analysis finds one main set of solar-like oscillations with unusually low amplitude. A second set of oscillations from the other star may exist, but this marginal detection is extremely faint. Because the two stars are nearly twins, KIC 9246715 is a difficult target for a precise test of the asteroseismic scaling relations, which yield $M = 2.17\pm0.13 \ M_{\odot}$ and $R = 8.26\pm0.16 \ R_{\odot}$. Both Star 1 and Star 2 are consistent with the inferred asteroseismic properties, but we suspect the oscillating star is Star 2, the slightly larger of the pair. \textbf{JUSTIFY WHY. Something about surface gravity and density.} We find evidence for stellar activity and modest tidal forces acting over the 171-day eccentric orbit, which are likely responsible for the essential lack of solar-like oscillations in one star and weak oscillations in the other. \revise{Mixed modes indicate the oscillating star is on the secondary red clump (core-He-burning), and stellar evolution modeling supports a coeval history for a pair of either red clump or red giant branch (shell-H-burning) stars.} This system is a useful case study and paves the way for a detailed analysis of more red giants in eclipsing binaries, an important benchmark for asteroseismology.
%\end{abstract}
  
  
  
  
  