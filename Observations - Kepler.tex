\section{Observations}\label{data}

\subsection{\emph{Kepler} light curves}\label{kepler}
Our light curves are from the \emph{Kepler} satellite in long-cadence mode (one data point every 29.4 minutes), and span 16 quarters---nearly four years---with only occasional gaps. Beyond successes in planet-hunting, \emph{Kepler} is incredibly useful for both red giant asteroseismology and studies of eclipsing binaries. Long-cadence observations are well-suited for red giant asteroseismology, as main sequence stars oscillate too rapidly to be seen. In addition, consistent target monitoring over several years has allowed \emph{Kepler} to discover numerous long-period eclipsing systems.

When studying long-period eclipsing binaries, slow drifts and discontinuities in the light curves can be dominated by instrumental effects. To remove these while preserving the astrophysically interesting signal (eclipse profiles, stellar variability, and oscillations), we employ a minimalistic detrending algorithm. EXPLAIN IT HERE.

%follow the approach in \citet{gau14}. We use simple aperture photometry, which is the integrated flux over each mask aperture without any pipeline processing. To combine all available quarters of light curve data into one time series, we first normalize the out-of-eclipse flux by dividing each quarter's data by the median flux with eclipses removed. To line up the ends of each ``chunk'' of the light curve, we proceed in one of two ways. When a gap is short with respect to photometric variability, we fit each side of the gap with a second order polynomial and extrapolate to the middle of the gap. The difference between both extrapolated values is then used to adjust the flux. When a gap is long with respect to photometric variability, we adjust the average flux of each ``chunk" so that they line up on either side of the gap. Following this process, the only apparent instrumental feature that remains is a periodic modulation corresponding to \emph{Kepler}'s 372.5-day Earth-trailing orbit. Because all the time series have gaps, it is not possible to use Fourier filtering to remove this signal. Instead, we subtract a 372.5-day period sine curve fit to the data plus its first harmonic, which reduces the amplitude of this modulation to less than 0.5\%.

% Need to get updated light curve through Q16 that is actually reduced using the method described and NOT just flattened out of eclipse.

% First figure: detrended light curve
%\begin{figure}[h!]
%\begin{center}
%\includegraphics[width=6 in]{fig1.eps}
%\end{center}
%\caption{Caption here.\label{keplerfig}}
%\end{figure}

