\section{Stellar atmosphere model}\label{atm}

\subsection{Spectral disentangling}\label{disentangle}
Before the two stars' atmospheres can be modeled, it is necessary to extract each star's spectrum from the observed binary spectra. While the location of a set of absorption lines in wavelength space is the only requirement for radial velocity studies, using an atmosphere model to measure $T_{\rm{eff}}$, $\log g$, and metallicity [Fe/H] for each star requires precise equivalent widths of particular absorption lines.

To accomplish this, we use the FDBinary tool \citep{ili04} on the spectral window 5320--7120 \AA. Following the approach in \citet{bec14}, we break the window into 180 pieces that each span about 10 \AA. FDBinary does not require a template spectrum, and instead uses the orbital parameters of a binary system to separate spectra in Fourier space. We tested FDBinary's capabilities by creating a set of fake double-lined spectra from a weighted sum of two identical spectra of Arcturus. When the orbital solution and flux ratio is correctly specified, the program returns a pair of single-lined spectra that are indistinguishable from the original.

FDBinary requires six parameters to define the shape of the radial velocity curve: orbital period, time of periastron passage (zeropoint), eccentricity, longitude of periastron, and amplitudes of each star's radial velocity curve. We set these to 171.277 days, 319.7 days\footnote{Units of BJD--2454833}, 0.35, 17.3 deg, 33.1 km s$^{-1}$, and 33.4 km s$^{-1}$, respectively. While FDBinary does include an optimization algorithm for any subset of these parameters, we use more robust fixed values from the photodynamical model described in Section \ref{model}. However, some of measured radial velocity points from Figure \ref{fig:rvfig} deviate as much as $\pm 2$ km s$^{-1}$ from the FDBinary model velocity curve. WHAT DO WE DO ABOUT THIS? ANYTHING?

FDBinary also requires the flux ratio of the two stars. WRITE MORE HERE.
%Because KIC 9246715 contains two similar red giants (though the oscillating giant is larger than its companion), we initially set the flux ratio to 1. However, as it became clear that the oscillating giant is nearly 2x as large as its companion, we estimated a flux ratio of 4:1. THIS IS ACTUALLY WRONG, FLUX RATIO IS CLOSER TO 1.
%To learn how this assumption affects the result, we further tested FDBinary by specifying an incorrect flux ratio for the set of fake double-lined spectra based on Arcturus. All of the correct absorption features were recovered, but one of the two output spectra had features that were too deep while the other's features were too shallow. Since this appears to systematically affect all absorption lines in the same way, we are confident that choosing a flux ratio that is correct within a factor of a few will not affect atmospheric modeling.

All 23 optical spectra of KIC 9246715 are processed together in FDBinary, and the result is a pair of disentangled spectra with zero radial velocity. These are shown in Figure \ref{fig:twospectra} with a characteristic ARCES spectrum taken near primary eclipse for comparison. This observed spectrum contains the signal of both stars.
