\section{Stellar atmosphere model}\label{atm}

\subsection{Spectral disentangling}\label{disentangle}
Before the two stars' atmospheres can be modeled, it is necessary to extract each star's spectrum from the observed binary spectra. While the location of a set of absorption lines in wavelength space is the only requirement for radial velocity studies, using an atmosphere model to measure $T_{\rm{eff}}$, $\log g$, and metallicity for each star requires precise equivalent widths of particular absorption lines.

To accomplish this, we use the FDBinary tool \citep{ili04} on the spectral window 5402--6750 \AA. Following the approach in \citet{bec14}, we break the window into 25 pieces that each span about 10 \AA. FDBinary does not require a template spectrum, and instead uses the orbital parameters of a binary system to separate spectra in Fourier space. We tested FDBinary's capabilities by creating a set of fake double-lined spectra from a weighted sum of two identical spectra of Arcturus. When the orbital solution and flux ratio is correctly specified, the program accurately returns a pair of single-lined spectra that are indistinguishable from the original.

FDBinary requires orbital period, time of periastron passage (zeropoint), eccentricity, longitude of periastron, and amplitudes of each star's radial velocity curve. We set these to 171.277 days, 321.189576 days\footnote{Units of BJD--2454833}, 0.36, 18 deg, 33.7 km s$^{-1}$, and 33.1 km s$^{-1}$, respectively. These orbital parameters come from a preliminary set of photodynamical models, which are later refined with stellar atmosphere models (see Section \ref{model}).

TO ADD: WHY WE DIDN'T USE FDBINARY'S FITTING ABILITY... IT SUCKS

FDBinary also requires the flux ratio of the two stars. Because KIC 9246715 contains two very similar red giants---the oscillating giant is slightly larger and cooler than its companion---we set the flux ratio to 1. To learn how this assumption affects the result, we further tested FDBinary by specifying an incorrect flux ratio for the set of fake double-lined spectra based on Arcturus. All of the correct absorption features were recovered, but one of the two output spectra had features that were systematically too deep while the other's features were systematically too shallow. Since this appears to affect all absorption lines in the same way, we are confident that choosing an incorrect flux ratio for two stars of similar brightness will not affect atmospheric modeling.

All 23 observed spectra of KIC 9246715 are processed together, and the result is a pair of disentangled spectra with zero radial velocity. These are shown in Figure \ref{fig:twospectra} with a characteristic ARCES spectrum taken near primary eclipse for comparison. This observed spectrum contains the signal of both stars.
