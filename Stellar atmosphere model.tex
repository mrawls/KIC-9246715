\section{Stellar atmosphere model}\label{atm}

\subsection{Spectral disentangling}\label{disentangle}
Before the two stars' atmospheres can be modeled, it is necessary to extract each star's spectrum from the observed binary spectra. While the location of a set of absorption lines in wavelength space is the only requirement for radial velocity studies, using an atmosphere model to measure $T_{\rm{eff}}$, $\log g$, and metallicity [Fe/H] for each star requires precise equivalent widths of particular absorption lines.

To this end, we use the FDBinary tool \citep{ili04} on the spectral window 5320--7130 \AA \ to perform spectral decomposition. Following the approach in \citet{bec14}, we break the window into 180 pieces that each span about 10 \AA. FDBinary does not require a template, and instead uses the orbital parameters of a binary to separate a set of double-lined spectral observations in Fourier space. We tested FDBinary's capabilities by creating a set of simulated double-lined spectra from a weighted sum of two identical spectra of Arcturus. When the orbital solution and flux ratio is correctly specified, the program returns a pair of single-lined spectra that are indistinguishable from the original.

FDBinary requires a set of double-lined spectral observations re-sampled evenly in $\ln \lambda$. For each input spectrum, it is important to apply barycentric corrections and subtract the binary's systemic velocity ($-4.48$ km s$^{-1}$ in this case). FDBinary further requires six parameters to define the shape of the radial velocity curve: orbital period, time of periastron passage (zeropoint), eccentricity, longitude of periastron, and amplitudes of each star's radial velocity curve. We set these to 171.277 days, 319.7 days\footnote{Units of BJD--2454833}, 0.35, 17.3 deg, 33.1 km s$^{-1}$, and 33.4 km s$^{-1}$, respectively. While FDBinary does include an optimization algorithm for any subset of these parameters, we use more robust fixed values from the photodynamical model described in Section \ref{model}. Finally, FDBinary requires a light ratio for each observation. Because the two stars are so similar, and none of our spectra were taken during eclipse, we set all light ratios to 1.

All 23 optical spectra of KIC 9246715 are processed together in FDBinary, and the result is a pair of disentangled spectra with zero radial velocity. A portion of the resulting individual spectra are shown in Figure \ref{fig:twospectra} with a characteristic ARCES spectrum containing signals from both stars and taken near primary eclipse for comparison.

    
    