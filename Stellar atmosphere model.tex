\section{Stellar atmosphere model}\label{atm}

\subsection{Spectral disentangling}\label{disentangle}
Before the two stars' atmospheres can be modeled, it is necessary to extract each star's spectrum from the observed binary spectrum. While the location of a set of absorption lines in wavelength space is the only requirement for radial velocity studies, using an atmosphere model to measure $T_{\rm{eff}}$, $\log g$, and metallicity for each star requires precise equivalent widths of particular absorption lines.

To accomplish this, we use the FDBinary tool \citep{ili04} on the spectral window 5402--6750 \AA. Following the approach in \citet{bec14}, we break the window into 26 pieces that each span about 10 \AA. FDBinary does not require a template spectrum, and instead uses the orbital parameters of a binary system to separate spectra in Fourier space. We use orbital parameters from a preliminary set of photodynamical models, which are later refined with stellar atmosphere models (see Section \ref{model}).

FDBinary requires orbital period, phase zeropoint, eccentricity, longitude of periastron, and amplitudes of each star's radial velocity curve. We set these to 171.277697 days, 5170.514777 days, 0.36, 18 deg, 34 km s$^{-1}$, and 34 km s$^{-1}$, respectively. FDBinary also requires the flux ratio of the two stars. Because KIC 9246715 contains two very similar red giants, we set this value to 1. All 23 observed spectra are processed together, and the result is the pair of disentangled spectra with zero radial velocity. These are shown in Figure \ref{twospectra} with a characteristic ARCES spectrum taken near primary eclipse for comparison. This observed spectrum contains the signal of both stars.
