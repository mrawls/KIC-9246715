\subsection{Searching for a second set of oscillations}

Given that the giants in KIC 9246715 are nearly twins, with $L_1/L_2 = 0.94$, $R_1/R_2 = 0.95$, $M_1/M_2 = 1.01$, and $T_1/T_2 = 1.01$, we test whether it is possible that we see only one set of oscillation modes because both stars are oscillating with virtually identical frequencies. It is unlikely that two sets of solar-like oscillations lie on top of one another, because the predicted difference in $\nu_{\rm{max}}$ for these not-quite-identical stars is about $11 \ \mu \rm{Hz}$ (from an inversion of Equations \ref{radeq} and \ref{masseq}), and the predicted difference in $\Delta \nu$ is about $0.7 \ \mu \rm{Hz}$. The effective frequency resolution of the power spectrum for four years of \emph{Kepler} data is about $0.001 \ \mu \rm{Hz}$, and, more importantly, the intrinsic observed mode line widths is about $0.5 \mu \rm{Hz}$.
% This is about five times as wide as normal RGs.
Given this, the oscillation pattern from a second star (if present) should \emph{not} appear to lie exactly on top of the oscillation pattern we do see. To illustrate this, the observed oscillation pattern in Figure \ref{fig:seismo} (red) is shown with predicted oscillation modes based on the physical parameters of both stars in the binary (OTHER COLOR(s)).

We also investigate whether the modes show any frequency modulation as a function of orbital phase by examining portions of the power spectrum spanning less than the orbital period. However, the solar-like oscillations are both weak and short-lived, so it is difficult to clearly resolve any modes in a power spectrum of a light curve segment less than several hundred days long. At $\nu_{\rm{max}} = 106 \ \mu \rm{Hz}$, the maximum frequency shift expected from a $\pm \ 30 \ \rm{km} \ \rm{s}^{-1}$ radial velocity is $0.01 \ \mu \rm{Hz}$. WHICH IS LESS THAN THE RESOLUTION? We conclude that one star truly lacks visible oscillations while the other has low-amplitude oscillations, and we are unable to definitively say which star is which.
    
    
    
    