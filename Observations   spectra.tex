\subsection{Ground-based spectroscopy}\label{spectra}
We took a total of 25 high-resolution echelle spectra from three ground-based spectrographs. At many orbital phases, prominent absorption lines show a clear double-lined signature when inspected by eye. We find that KIC 9246715 is an excellent target for obtaining radial velocity curves for both stars in the binary. A long time span of observations was necessary due to the 171.277-day orbital period of the binary and visibility of the \emph{Kepler} field from the observing sites.

\subsubsection{TRES echelle at FLWO}\label{tres}
We obtained 13 high-resolution spectra from the Fred Lawrence Whipple Observatory (FLWO) 1.5-m telescope in Arizona using the Tillinghast Reflector Echelle Spectrograph (TRES) from 2012 March through 2013 April. The wavelength range for TRES is 3900--9100 \AA, and the average resolution is $R \sim 30,000$. We reduced the data using the freely available standard pipeline\footnote{http://tdc-www.harvard.edu/instruments/tres/reduce.html}.

\subsubsection{ARCES echelle at APO}\label{arces}
We obtained ten high-resolution spectra from the Apache Point Observatory (APO) 3.5-m telescope in New Mexico using the Astrophysical Research Consortium Echelle Spectrograph (ARCES) from 2012 June through 2013 September. The wavelength range for ARCES is 3200--10,000 \AA \ with no gaps, and the average resolution is $R \sim 31,000$. We reduced the data using a standard echelle reduction techniques and Karen Kinemuchi's ARCES cookbook (private communication)\footnote{http://astronomy.nmsu.edu:8000/apo-wiki/wiki/ARCES#reduction - ARC Echelle Spectragraph (ARCES) Data Reduction Cookbook}. We found a minor discrepancy ($\sim 0.3$ \AA \ at $7640$ \AA) between different nights' wavelength solutions for some of the earlier spectra, and used telluric features in this wavelength regime to apply small shifts in velocity space when necessary. We subsequently took thorium-argon calibration spectra more frequently which has corrected the problem.

\subsubsection{APOGEE spectra}\label{apogee}
Two observations. Coming soon.
