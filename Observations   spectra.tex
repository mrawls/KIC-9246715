\subsection{Ground-based spectroscopy}\label{spectra}
We have a total of 25 high-resolution spectra from three spectrographs. At many orbital phases, prominent absorption lines show a clear double-lined signature when inspected by eye. We find that KIC 9246715 is an excellent target for obtaining radial velocity curves for both stars in the binary as the stellar flux ratio is close to unity. A long time span of observations was necessary due to the 171.277-day orbital period of the binary and visibility of the \emph{Kepler} field from the observing sites.

\subsubsection{TRES echelle from FLWO}\label{tres}
We obtained 13 high-resolution optical spectra from the Fred Lawrence Whipple Observatory (FLWO) 1.5-m telescope in Arizona using the Tillinghast Reflector Echelle Spectrograph (TRES) from 2012 March through 2013 April. The wavelength range for TRES is 3900--9100 \AA, and the average resolution is 30,000. We reduced the data using the freely available standard pipeline\footnote{\url{http://tdc-www.harvard.edu/instruments/tres/reduce.html}}.

\subsubsection{ARCES echelle from APO}\label{arces}
We also obtained ten high-resolution optical spectra from the Apache Point Observatory (APO) 3.5-m telescope in New Mexico using the Astrophysical Research Consortium Echelle Spectrograph (ARCES) from 2012 June through 2013 September. The wavelength range for ARCES is 3200--10,000 \AA \ with no gaps, and the average resolution is 31,000. We reduced the data using standard echelle reduction techniques and Karen Kinemuchi's ARCES cookbook (private communication)\footnote{\url{http://astronomy.nmsu.edu:8000/apo-wiki/wiki/ARCES\#reduction} - ARC Echelle Spectragraph (ARCES) Data Reduction Cookbook}.
%We found a minor discrepancy ($\sim 0.3$ \AA \ at $7640$ \AA) between different nights' wavelength solutions for some of the earlier spectra, and used telluric features in this wavelength regime to apply small shifts in velocity space when necessary. We subsequently took thorium-argon calibration spectra more frequently which has corrected the problem.
% ^^ the above isn't relevant anymore now that we have section wavelength, below.

\subsubsection{APOGEE spectra from APO}\label{apogee}
We finally obtained two near-IR spectra of KIC 9246715 from the Sloan Digital Sky Survey-III (SDSS-III) Apache Point Observatory Galactic Evolution Experiment (APOGEE) survey \citep{2015arXiv150100963A}. The wavelength range for APOGEE is 1.5--1.7 $\mu$m with a nominal resolution of 22,500. The pair of spectra were reduced with the standard APOGEE pipeline, but not combined.

\subsubsection{Global wavelength solution}\label{wavelength}
Because the observations come from three different spectrographs at two different observatory sites, it is critical to apply a consistent wavelength solution that yields the same radial velocity zeropoint for all observations. This zeropoint is a function of the atmospheric conditions at the observatory and the instrument being used. Typically such a correction can be done with RV standard stars after a wavelength solution has been applied based on ThAr lamp observations. However, we lacked RV standard star observations, and some of the earlier ARCES observations had insufficiently frequent ThAr calibration images to arrive at a reliable wavelength solution. (We subsequently took ThAr images more frequently to address the latter issue.) To arrive at a consistent velocity zeropoint for all spectra, we used TelFit \citep{gul14} to generate a telluric line model of the O2 A-band (7595--7638 \AA) with $R = 31,000$ at STP. We then shifted the ARCES and TRES spectra in velocity space using the broadening function technique (see Section \ref{bf}) so they all line up with the TelFit model. The shifts ranged from $-0.88 - 2.18$ km s$^{-1}$, with the majority having a magnitude $< 0.3$ km s$^{-1}$.
%APOGEE spectra are fiber-fed and reported in vacuum wavelengths, TRES spectra are fiber-fed and typically calibrated to CfA-standard air wavelengths, and ARCES spectra are slit-based and typically reported in local air wavelengths.
  
  
  
  