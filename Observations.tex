\section{Observations}\label{data}

\subsection{\emph{Kepler} light curves}\label{kepler}
Our light curves are from the \emph{Kepler} satellite in long-cadence mode (one data point every 29.4 minutes), and span 16 quarters---nearly four years---with only occasional gaps. Beyond successes in planet-hunting, \emph{Kepler} is incredibly useful for both red giant asteroseismology and studies of eclipsing binaries. Long-cadence observations are well-suited for red giant asteroseismology, as main sequence stars oscillate too rapidly to be seen. In addition, consistent target monitoring over several years has allowed \emph{Kepler} to discover numerous long-period eclipsing systems.

When studying long-period eclipsing binaries, slow drifts and discontinuities in the light curves can be dominated by instrumental effects. To remove these while preserving the astrophysically interesting signal (eclipse profiles, stellar variability, and oscillations), we employ a minimalistic detrending algorithm. EXPLAIN IT HERE.

%follow the approach in \citet{gau14}. We use simple aperture photometry, which is the integrated flux over each mask aperture without any pipeline processing. To combine all available quarters of light curve data into one time series, we first normalize the out-of-eclipse flux by dividing each quarter's data by the median flux with eclipses removed. To line up the ends of each ``chunk'' of the light curve, we proceed in one of two ways. When a gap is short with respect to photometric variability, we fit each side of the gap with a second order polynomial and extrapolate to the middle of the gap. The difference between both extrapolated values is then used to adjust the flux. When a gap is long with respect to photometric variability, we adjust the average flux of each ``chunk" so that they line up on either side of the gap. Following this process, the only apparent instrumental feature that remains is a periodic modulation corresponding to \emph{Kepler}'s 372.5-day Earth-trailing orbit. Because all the time series have gaps, it is not possible to use Fourier filtering to remove this signal. Instead, we subtract a 372.5-day period sine curve fit to the data plus its first harmonic, which reduces the amplitude of this modulation to less than 0.5\%.

% Need to get updated light curve through Q16 that is actually reduced using the method described and NOT just flattened out of eclipse.

% First figure: detrended light curve
%\begin{figure}[h!]
%\begin{center}
%\includegraphics[width=6 in]{fig1.eps}
%\end{center}
%\caption{Caption here.\label{keplerfig}}
%\end{figure}

\subsection{Ground-based spectroscopy}\label{spectra}
% Meredith is working on this section
We took a total of 23 high-resolution echelle spectra from two ground-based observatories. At many orbital phases, prominent absorption lines show a clear double-lined signature when inspected by eye. We find that KIC 9246715 is an excellent target for obtaining radial velocity curves for both stars in the binary.

	\subsubsection{TRES echelle at FLWO}\label{tres}
	% Meredith will write this part
We obtained 13 high-resolution spectra from the Fred Lawrence Whipple Observatory (FLWO) 1.5 m telescope in Arizona using the Tillinghast Reflector Echelle Spectrograph (TRES) from 2012 March through 2013 April. The wavelength range for TRES is RANGE HERE and the resolution is RESOLUTION HERE. We reduced the data using the freely available standard pipeline: BRIEFLY LIST STEPS HERE.

	\subsubsection{ARCES echelle at APO}\label{arces}
We obtained ten high-resolution spectra from the Apache Point Observatory (APO) 3.5 m telescope in New Mexico using the Astrophysical Research Consortium Echelle Spectrograph (ARCES) from 2012 June through 2013 September. A long time span was necessary due to the 171.277-day orbital period of the binary and visibility of the Kepler field from APO. The wavelength range for ARCES is 3200--10,000 \AA \ with no gaps, and the resolution is $R \sim 31,000$. We reduced the data using a standard pipeline: BRIEFLY LIST STEPS HERE. We found a minor discrepancy ($\sim 0.3$ \AA \ at $7640$ \AA) between different nights' wavelength solutions for some of the earlier spectra, and used telluric features in this wavelength regime to apply small shifts in velocity space when necessary. We subsequently took ThAr calibration images more frequently which seems to have corrected the problem.
