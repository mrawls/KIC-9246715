\subsection{Tidal forces and stellar evolution}\label{tides}

To estimate how tidal forces may affect the orbital eccentricity, we follow the approach of \citet{ver95}, a study of tidal circularization in binaries containing giant stars. We consider a detached binary with constant $a$ and constant binary masses. Equations 5 and 6 \citep{ver95} state:
[YOU NEED TO INTRODUCE BETTER THE PAPER BY VERBUNT ET AL: DEFINE THE PARAMETERS LIKE THE ECCENTRICITY VARIATION DLNE, THE INTEGRAL TERM. NOT UNDERSTANDABLE AS IT IS].

\begin{equation}\label{tide1}
\Delta \ln e = -1.7 \times 10^{-5} \ f \ {\left( \frac{M}{M_{\odot}} \right)}^{-11/3} \ q(1+q)^{-5/3} \ I(t) {\left( \frac{P_{\rm{orb}}}{\rm{day}} \right)}^{-16/3},
\end{equation}

where $f$ is a factor of order unity, $q$ is the ratio of the companion star's mass $M_2$ to the primary star's mass $M$, and

\begin{equation}
I(t) \equiv \int_0^t \left( \frac{T_{\rm{eff}}(t')}{4500 \rm{K}} \right)^{4/3} \ \left( \frac{M_{\rm{env}}(t')}{M_{\odot}} \right)^{2/3} \ \left( \frac{R(t')}{R_{\odot}} \right)^8 \ dt',
\end{equation}

where $M_{\rm{env}}$ is the mass of the convective envelope of the primary star. It is important to note that \citet{ver95} assumed circularization would proceed by a small secondary star (main sequence or white dwarf) imposing an equilibrium on a large giant, while the situation with KIC 9246715 is more complicated. For a thorough review of tidal forces in stars, see \citet{ogi14}.

For Star 2 acting on Star 1, we find RESULT FROM CALCULATION HERE. For Star 1 acting on Star 2, we find RESULT FROM CALCULATION HERE. In either case, this means that the change in eccentricity over the binary's lifetime should theoretically be greater than 1. In practical terms, this means that regardless of KIC 9246715's original eccentricity, it should have evolved to zero by the present day, when both stars are in the red clump phase. The observed $e = 0.35$ is inconsistent. OK SO WHY.

Tidal forces also tend to synchronize a binary star's orbit with the stellar rotation period, generally on shorter timescales than required for circularization. \citet{gau14} measure a period of variability from the power spectrum of KIC 9246715 that is very nearly half the orbital period. They interpret this as the starspot modulation, a proxy for rotation period. TALK ABOUT SURFACE VERSUS CORE ROTATION? Synchronization? VSINI? (P.G. 30.6.15: WE DO NOT HAVE RELIABLE CLUES ON CORE ROTATION SO FAR. WHAT BENOIT DID ASSUMES SURFACE ROATION IS MUCH SLOWER THAN CORE. NOT THE CASE HERE) 

\textit{(To calculate $I$, use MESA models from many points, and maybe make a plot like Verbunt's Fig 1 as a sanity check. Then, add 0.35 to whatever $\Delta \ln e$ is for the star's lifetime to get starting eccentricity. We think this will be $>1$ but we need to check. Finally, look at Verbunt Fig 4 and figure out where this system would roughly fall. Consistent with other binaries?)}

INTRODUCE AND DISCUSS THE AGE VS MASS MESA MODEL PLOT THING.
THE MODES ARE SOME 5X BROADER THAN TYPICAL RED GIANTS. OK SO?

MAKE THIS INTO ACTUAL TEXT:

Centrifugal forces dissipate energy that would cause circularization. Causes individual rotations to synchronize (usually synchronization happens first, and circularization happens second). Neither of those things have happened here, but it seems like the rotation knows about the orbit, so SOMETHING has happened... just not synchronization. This should have happened even faster since it's two RGs, not a RG + MS (or WD). The fact that both of them have convection zones that could be causing this centrifugal force situation means that this should have happened a long time ago in a hand-wavey sort of way. CAN WE EVEN SAY FOR SURE THAT WE HAVE A DECENT MEASURE OF THE INTERNAL ROTATION FROM ASTEROSEISMOLOGY TO COMPARE TO VSINI (SEE PATRICK'S COMMENT ABOVE)?

The fact that only one star is oscillating points to how tidal activity may affect oscillations.
Important for other people because over half of cool stars should be in binaries! Using RGs to probe the galaxy has to be done carefully because of external influences of binarity on oscillations.
  