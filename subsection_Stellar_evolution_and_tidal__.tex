\subsection{Stellar evolution and tidal forces}\label{tides}

Over the course of KIC 9246715's life, both stars have evolved in tandem to reach the configuration we see today. We quantify this with simple stellar evolution models created using the Modules for Experiments in Stellar Astrophysics (MESA) code \citep{pax11,pax13,pax15}. Figure \ref{fig:mesa} presents a suite of models with various initial stellar masses. All the models include overshooting and assume no mass loss. Recall that the masses and radii of each star are $M_1 = 2.16 \pm 0.04\ M_{\odot}$, $M_2 = 2.14 \pm 0.03\ M_{\odot}$, $R_1 = 7.90 \pm 0.04 \ R_{\odot}$, $R_2 = 8.33 \pm 0.04 \ R_{\odot}$. The stage of each star's life as it ages in Figure \ref{fig:mesa} is color-coded, and two lines of constant radii (corresponding to Star 1 and Star 2) are shown. From this, we conclude that both stars are indeed coeval, are both presently on the red giant branch, and are too small by $\sim 1 \ R_\odot$ to have evolved past this stage onto the red clump.

We explore several possibilities to explain why these stars' nearly identical evolutionary histories appear to place them on the red giant branch while the asteroseismic period spacing suggests they are on the red clump:
\begin{itemize}
\item Adding a prescription for moderate red-giant-branch mass loss ($\eta = 0.4$, see \citet{mig12}) to the MESA model did not appreciably change stellar radius as a function of evolutionary stage. Even a more extreme mass-loss rate ($\eta = 0.7$) did not significantly affect the radii, essentially because the star is too low-mass to lose much mass.
\item If we increase the mixing-length parameter $\alpha$ from the standard solar value of 2 to 3 in the MESA model, effectively increasing the efficiency of convection, it is possible to produce a red clump star small enough to agree with our two measured radii. Given that low-amplitude solar-like oscillations are indicative of physically different conditions in the convection zone (presumably due to magnetic activity), this is not unreasonable.
%If these physical differences are significant enough, stellar evolution may proceed differently than we expect, and it may be possible to create red clump stars with smaller radii.
\item The noisy period spacing estimate ($\Delta \Pi \simeq 150 \ \rm{sec}$) may not be measuring what we expect due to rotational splitting of mixed oscillation modes. If the true period spacing is closer to $\Delta \Pi \simeq 80 \ \rm{sec}$, this would explain the disagreement. A detailed discussion of rotational splitting behavior in slowly rotating red giants is beyond the scope of this paper, but is explored in \citet{gou13}.
\end{itemize}

  