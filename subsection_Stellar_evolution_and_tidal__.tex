\subsection{Stellar evolution and tidal forces}\label{tides}
Over the course of KIC 9246715's life, both stars have evolved in tandem to reach the configuration we see today. We quantify this with simple stellar evolution models created using the Modules for Experiments in Stellar Astrophysics (MESA) code \citep{pax11,pax13,pax15}. Figure \ref{fig:mesa} presents a suite of models with various initial stellar masses. All the models include overshooting for all the convective zone boundaries with an efficiency of $f = 0.016$ \citep{her00} and assume no mass loss. The stage of each model star's life as it ages in Figure \ref{fig:mesa} is color-coded. Two vertical lines of constant mass (corresponding to $M_1$ and $M_2$) and two curved lines of constant radii (corresponding to $R_1$ and $R_2$) are shown. In general, coeval stars on the red giant branch must have masses within $1\%$ of each other, whereas there is slightly more freedom on the horizontal branch due to its longer evolutionary lifetime. The uncertainties in $M_1$, $M_2$, and $M_2/M_1$ do allow the stars to be equal in mass.

Stellar evolution modeling with MESA using standard input parametrization places both stars on the red giant branch. Specifically, the radii of Star 1 and Star 2 are too small by at least $0.2 \ R_\odot$ compared to the MESA models to have evolved past this stage onto the red clump. We consider several ideas as to why these models and the asteroseismic period spacing are in apparent contradiction:
\begin{itemize}
\item Mass loss: Adding a prescription for moderate red-giant-branch mass loss ($\eta = 0.4$, see \citealt{mig12}) to the MESA model does not appreciably change stellar radius as a function of evolutionary stage. Even a more extreme mass-loss rate ($\eta = 0.7$) does not significantly affect the radii, essentially because the star is too low-mass to lose much mass.
\item He abundance: Increasing the initial He fraction in the MESA model does not allow for smaller stars in the red clump phase, because shell-H burning is very efficient with additional He present. As a result, the star maintains a high luminosity and therefore a larger radius as it evolves from the tip of the red giant branch to the red clump.
\item Convective overshoot: The MESA models in this work assume a reasonable overshoot efficiency as described above ($f = 0.016$). We can barely make a red clump as small as $R_2 \simeq 8.3 \ R_\odot$ by adjusting this value between 0--0.03. With less overshoot, the RGB phase as shown in Figure \ref{fig:mesa} increases in duration, more readily allowing stars with $M_1 = 2.16 \ M_\odot$ and $M_2 = 2.14 \ M_\odot$ to be coeval.
\item Mixing length: Increasing the mixing-length parameter $\alpha$ from the standard solar value of 2 to 3 in the MESA model, which effectively increases the efficiency of convection, does produce a red clump star small enough to agree with both measured radii. This is because it reduces the temperature gradient in the near-surface layers, increasing the effective temperature while reducing the radius at constant luminosity. 
\item Period spacing: The noisy period spacing estimate ($\Delta \Pi \simeq 150 \ \rm{sec}$) may not be measuring what we expect due to possible rotational splitting of mixed oscillation modes. If the true period spacing is closer to $\Delta \Pi \simeq 80 \ \rm{sec}$, this would explain the disagreement. A detailed discussion of rotational splitting behavior in slowly rotating red giants is explored in \citet{gou13}.
\end{itemize}

  
  