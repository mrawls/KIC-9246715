\subsection{Stellar evolution and tidal forces}\label{tides}
Over the course of KIC 9246715's life, both stars have evolved in tandem to reach the configuration we see today. We quantify this with simple stellar evolution models created using the Modules for Experiments in Stellar Astrophysics (MESA) code \citep{pax11,pax13,pax15}. Figure \ref{fig:mesa} presents a suite of models with various initial stellar masses. All the models include overshooting for all the convective zone boundaries with an efficiency of $f = 0.016$ \citep{her00}, assume no mass loss, \revise{and set the mixing-length parameter $\alpha = 2.5$. The standard solar value of $\alpha = 2$ does not allow for sufficiently small stars beyond the red giant branch}. The stage of each model star's life as it ages in Figure \ref{fig:mesa} is color-coded, and \revise{curved lines of constant radii corresponding to $R_1 \pm \sigma$ (gray) and $R_2 \pm \sigma$ (white), within the ranges of $M_1 \pm \sigma$ and $M_2 \pm \sigma$, respectively, are shown. There are two instances in each pair of model stars' lives when they have the same radii as the stars in KIC 9246715: once on the red giant branch, and again on the secondary red clump (horizontal branch).}

In general, coeval stars on the red giant branch must have masses within $1\%$ of each other, whereas there is slightly more freedom on the horizontal branch due to its longer evolutionary lifetime. \revise{Both model stars in Figure \ref{fig:mesa} can be the same age on the horizontal branch, but not on the red giant branch. Without $\alpha > 2$, the MESA model stars on the horizontal branch are always larger than those in KIC 9246715. We consider several ideas as to why the MESA models and the evolutionary stage determined from asteroseismic mixed-mode period spacing in Section \ref{subsubsec_mixed} may differ:}
\begin{itemize}
\item Mass loss: Adding a prescription for moderate red-giant-branch mass loss ($\eta = 0.4$, see \citealt{mig12}) to the MESA model does not appreciably change stellar radius as a function of evolutionary stage. Even a more extreme mass-loss rate ($\eta = 0.7$) does not significantly affect the radii, essentially because the star is too low-mass to lose much mass.
\item He abundance: Increasing the initial He fraction in the MESA model does not allow for smaller stars in the red clump phase, because shell-H burning is very efficient with additional He present. As a result, the star maintains a high luminosity and therefore a larger radius as it evolves from the tip of the red giant branch to the red clump.
\item Convective overshoot: The MESA models in this work assume a reasonable overshoot efficiency as described above ($f = 0.016$). We tried varying this from 0--0.03, and can barely make a red clump star as small as $8.3 \ R_\odot$ when $f = 0.01$. With less overshoot, the RGB phase as shown in Figure \ref{fig:mesa} increases in duration, which allows a higher probability for stars of $M_1$ and $M_2$ to both be on the RGB.
\item Period spacing: The period spacing \revise{$\Delta \Pi_1 = 150.4 \pm 1.4 \ \rm{s}$} may not be measuring what we expect due to rotational splitting of mixed oscillation modes. If the true period spacing is closer to $\Delta \Pi_1 \simeq 80 \ \rm{sec}$, \revise{this would put the oscillating star on the red giant branch. However, as demonstrated in Section \ref{subsubsec_mixed}, the mixed modes do agree best with a secondary red clump star.} A detailed discussion of rotational splitting behavior in slowly rotating red giants is explored in \citet{gou13}.
\item Mixing length: \revise{As discussed above,} increasing the mixing-length parameter from the standard solar value of $\alpha = 2$ to $\alpha = 2.5$ in the MESA model, which effectively increases the efficiency of convection, produces a red clump star small enough to agree with both measured radii. This is because it reduces the temperature gradient in the near-surface layers, increasing the effective temperature while reducing the radius at constant luminosity. \revise{This is what we employ to make horizontal branch stars that agree with $R_1$ and $R_2$.}
\end{itemize}
