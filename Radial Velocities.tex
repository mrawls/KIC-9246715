\section{Radial velocities}\label{rvs}
To extract radial velocities from the spectra, we use the broadening function (BF) technique as outlined by \citet{ruc02}. The BF is a function that transforms a sharp-line spectrum of a standard star into a Doppler-broadened binary spectrum. The BF convolution is solved with singular value decomposition. This technique is preferred over the more familiar cross-correlation function (CCF), because the BF is a true linear deconvolution while the CCF is a non-linear proxy and is less suitable for double-lined spectra.

%We use a spectrum of Arcturus taken with ARCES in 2004 as a BF template for the optical spectra,
We use a PHOENIX model spectrum CITATION HERE of a star with PHYSICAL PARAMETERS HERE as a BF template, interpolated to have a similar resolution as the spectrographs used. For the optical spectra we consider the wavelength range 5400--6700 \AA. This region is chosen because it has a high signal-to-noise ratio and minimal telluric features. For the near-IR APOGEE spectra, we consider the wavelength range VALUES HERE. We first smooth the BF with a Gaussian to remove un-correlated, small-scale noise below the size of the spectrograph slit, and then fit Gaussian profiles to measure the location of the BF peaks in velocity space. The geocentric (raw) results from the BF technique are shown for each observation in Figure \ref{fig:bffig}. The final derived radial velocity curve with barycentric corrections is shown in Figure \ref{fig:rvfig}.
%Radial velocities for the two near-IR APOGEE spectra are obtained in the same way, using an APOGEE spectrum of Arcturus taken in 2014 as the BF template.
