\section{Radial velocities}\label{rvs}
\subsection{The broadening function}\label{bf}
To extract radial velocities from the spectra, we use the broadening function (BF) technique as outlined by \citet{ruc02}. The BF is a function that transforms a sharp-line spectrum of a standard star into a Doppler-broadened spectrum. In practice, it can be used to describe any deviation of an observed spectrum from an idealized sharp-line spectrum: various forms of line broadening, shifted lines due to Doppler radial velocity shifts, two sets of lines in the case of a spectroscopic binary, etc. The BF convolution is solved with singular value decomposition as described in \citet{ruc02}. This technique is generally preferred over the more familiar cross-correlation function (CCF), because the BF is a true linear deconvolution while the CCF is a non-linear proxy and is less suitable for double-lined spectra. For this analysis, we adapted the IDL routines provided by Rucinski\footnote{\url{http://www.astro.utoronto.ca/~rucinski/SVDcookbook.html}} into python\footnote{\url{https://github.com/mrawls/BF-rvplotter}}.

We use a PHOENIX BT-Settl model atmosphere spectrum as a BF template \citep{all03}. This particular model uses \citet{Asplund_2009} solar abundance values for a star with $T_\rm{eff} = 4800$ K, $\log g = 2.5$, and solar metallicity. Since the BF handles line broadening between template and target robustly, we do not adjust the resolution of the template. A model spectrum was used instead of an actual star to avoid inconsistencies between the optical and IR regime (this model can span from 1--9995000 \AA), additional barycentric corrections, spurious telluric line BF peaks, and uncertainties in a template star's systemic RV. However, when we tested the BF with an observation of Arcturus as a template, the peaks were narrower and had larger amplitudes. The advantages that come from using an observed stellar spectrum as a template may be critical for future work, especially when measuring the RVs of main-sequence companions which contribute only a few percent of the flux to an RG/EB.
% Are the peaks broader because it's part of the BROADENING function solution, i.e., is it showing the discrepancy in resolution?!

For the optical spectra we consider the wavelength range 5400--6700 \AA. This region is chosen because it has a high signal-to-noise ratio and minimal telluric features. For the near-IR APOGEE spectra, we consider the wavelength range 15150--16950 \AA. We smooth the BF with a Gaussian to remove un-correlated, small-scale noise below the size of the spectrograph slit, and then fit Gaussian profiles to measure the location of the BF peaks in velocity space. The geocentric (uncorrected) results from the BF technique are shown for the optical spectra in Figure \ref{fig:bffig}. The results look similar for the near-IR spectra. The final derived radial velocity curve with barycentric corrections is shown in Figure \ref{fig:rvfig}.

\subsection{Comparison with TODCOR}\label{todcor}
To confirm that the BF-extracted radial velocities are accurate, we also use TODCOR \citep{zuc94} to extract RVs for the TRES spectra. TODCOR, which stands for two-dimensional cross-correlation, uses a template spectrum from a library with a narrow spectral range (5050--5350 \AA) to make a two-component RV curve for spectroscopic binaries. It is commonly used with TRES spectra for eclipsing binary studies. From the RV curve, TODCOR subsequently calculates an orbital solution. We use the full TODCOR RV extractor + orbital solution calculator for the TRES spectra, and compare this with the TODCOR orbital solution calculator for the combined ARCES, TRES, and APOGEE RV points which were extracted with the BF technique. We find that the two orbital solutions are in excellent agreement.
