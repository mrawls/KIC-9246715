\section{Radial velocities}\label{rvs}
To extract radial velocities from the spectra, we use the broadening function (BF) technique as outlined by \citet{ruc02}. This is generally preferred over the more familiar cross-correlation function (CCF), because the BF is a true linear deconvolution while the CCF is a non-linear proxy and therefore less suitable for double-lined (SB2) spectra. We use a spectrum of Arcturus taken with ARCES in 2004 as a template for the BF, and consider the wavelength range 5400--6700 \AA. This region is chosen because it has a high signal-to-noise ratio and minimal telluric features. We first smooth the BF with a Gaussian (total area = 1, $\sigma=1.5$) to remove un-correlated, small-scale noise below the size of the spectrograph slit, and then fit Gaussian profiles to measure the location of the BF peaks in velocity space. The results from the BF technique are shown for each observation in Figure \ref{fig:bffig}. The final derived radial velocity curve with barycentric corrections is shown in Figure \ref{fig:rvfig}.
