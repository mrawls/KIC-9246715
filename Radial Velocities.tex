\section{Radial velocities}\label{rvs}
\subsection{The broadening function}\label{bf}
To extract radial velocities from the spectra, we use the broadening function (BF) technique as outlined by \citet{ruc02}. In the simplest terms, the BF is a function that transforms any sharp-line spectrum into a Doppler-broadened spectrum. The BF technique involves solving a convolution equation for the Doppler broadening kernel $B$, $P(x) = \int B(x') T(x - x') dx'$, where $P$ is an observed spectrum of a binary and $T$ is a spectral template spanning the same wavelength window \citep{ruc14}. In practice, the BF can be used to characterize any deviation of an observed spectrum from an idealized sharp-line spectrum: various forms of line broadening, shifted lines due to Doppler radial velocity shifts, two sets of lines in the case of a spectroscopic binary, etc. The BF deconvolution is solved with singular value decomposition as described in \citet{ruc02}. This technique is generally preferred over the more familiar cross-correlation function (CCF), because the BF is a true linear deconvolution while the CCF is a non-linear proxy and is less suitable for double-lined spectra. The BF technique normalizes the result so that the velocity integral $\int B(v) dv = 1$ for an exact spectral match of the observed and template spectra. For this analysis, we adapted the IDL routines provided by Rucinski\footnote{\url{http://www.astro.utoronto.ca/~rucinski/SVDcookbook.html}} into python\footnote{\url{https://github.com/mrawls/BF-rvplotter}}.

We use a PHOENIX BT-Settl model atmosphere spectrum as a BF template \citep{all03}. This particular model uses \citet{asp09} solar abundance values for a star with $T_{\rm{eff}} = 4800$ K, $\log g = 2.5$ {\bf{(YOU SHOULD PUT 3.0 IF THE DIFFERENCE WAS NEGLIGIBLE. TO BE COHERENT WITH WHAT WE FIND... \textit{MR: there is a footnote to this effect, hover over the 3 below})}}, and solar metallicity, selected based on revised KIC values for KIC 9246715 \citep{hub14.2}\footnote{We later confirm that the RV results are indistinguishable from those measured with a more accurate BF model template ($T_{\rm{eff}} = 5000$ K, $\log g = 3.0$, see Table \ref{table2}).}. Since the BF handles line broadening between template and target robustly, we do not adjust the resolution of the template. For comparison, we test the BF with an observation of Arcturus as a template, and find the peaks are narrower and have larger amplitudes. These qualities may be essential to measure RVs in the situation where a companion star is extremely faint, but the RVs of both stars in KIC 9246715 appear with similar strength, so we choose a model atmosphere template for simplicity. Specifically, using a model avoids inconsistencies between the optical and IR regime, additional barycentric corrections, spurious telluric line peaks, and uncertainties from a template star's systemic RV. {\bf{THIS STATEMENT IS UNCLEAR, YOUR SENTENCE DOES NOT ADEQUATLY EXPLAIN WHY USING A TRUE SPECTRUM WORKS BETTER WITH A FAINT COMPANION:}} However, the advantages of using a real star spectrum as a BF template instead of a model will likely be crucial for future work, as most other RG/EBs are composed of a bright RG and relatively faint main sequence companion.

For the optical spectra we consider the wavelength range 5400--6700 \AA. This region is chosen because it has a high signal-to-noise ratio and minimal telluric features. For the near-IR APOGEE spectra, we consider the wavelength range 15150--16950 \AA. We smooth the BF with a Gaussian to remove un-correlated, small-scale noise below the size of the spectrograph slit, and then fit Gaussian profiles to measure the location of the BF peaks in velocity space. The geocentric (uncorrected) results from the BF technique are shown for the optical spectra in Figure \ref{fig:bffig}. The results look similar for the near-IR spectra. The final derived radial velocity points with barycentric corrections are presented in Table \ref{table0} and Figure \ref{fig:rvfig}.

\subsection{Comparison with TODCOR}\label{todcor}
To confirm that the BF-extracted radial velocities are accurate, we also use TODCOR \citep{zuc94} to extract radial velocities for the TRES spectra. TODCOR, which stands for two-dimensional cross-correlation, uses a template spectrum from a library with a narrow spectral range (5050--5350 \AA) to make a two-component radial velocity curve for spectroscopic binaries. It is commonly used with TRES spectra for eclipsing binary studies. From the radial velocity curve, TODCOR subsequently calculates an orbital solution. We use the full TODCOR RV extractor + orbital solution calculator for the TRES spectra, and compare this with the TODCOR orbital solution calculator for the combined ARCES, TRES, and APOGEE RV points which were extracted with the BF technique. We find that the two orbital solutions are in excellent agreement. The TODCOR RVs (available for TRES spectra only) are on average $0.22 \pm 0.25 \ \rm{km \ s}^{-1}$ systematically lower than the BF RVs, which we attribute to a physically unimportant difference in RV zeropoint.
 
  