\subsection{A hint of a second set of oscillations}
\label{subsec_second_osc}
Given that the giants in KIC 9246715 are nearly twins, we test whether it is possible that we see only one set of oscillation modes because both stars are oscillating with virtually identical frequencies. It is unlikely that two sets of solar-like oscillations lie on top of one another, because the predicted $\nu_{\rm{max}}$ for these not-quite-identical stars are $115.47 \pm 4.46$ and $102.9 \pm 3.46 \ \mu\rm{Hz}$ for Star 1 and Star 2 respectively (from an inversion of Equations \ref{density} and \ref{gravity}), and the predicted $\Delta\nu_{\rm{obs}}$ are $8.85 \pm 0.15$ and $8.14 \pm 0.12 \ \mu\rm{Hz}$ for Star 1 and Star 2 respectively. The effective frequency resolution of the power spectrum for four years of \emph{Kepler} data is about $0.008 \ \mu \rm{Hz}$, and, more importantly, the intrinsic observed mode line widths is about $0.5 \ \mu \rm{Hz}$. Given this, the oscillation pattern from a second star (if present) should \emph{not} appear to lie exactly on top of the oscillation pattern we do see.

Searching for a second set of oscillations is motivated by the broad, mixed-mode-like appearance of the $\ell=0$ modes in Figure \ref{fig:echelle}, where mixed modes are not physically possible, and by the faint diagonal structure mostly present on the upper left side of the $\ell=1$ mode ridge. Even though oscillation modes from the two stars should not perfectly overlap, modes of degree $\ell=0,1$ of one star can almost overlap modes of degree $\ell=1,0$ of the other star.

% Benoit
\revise{The universal red giant oscillation pattern \citep{mos11} yields $\Delta \nu =  8.31 \pm 0.04 \ \mu\rm{Hz}$ for this system (Section \ref{subsubsec_main_osc}), but also shows many extra peaks which cannot be mixed modes. We therefore test the hypothesis of a binary companion. The universal oscillation pattern allows us to tentatively allocate the extra peaks to a pressure-mode oscillation pattern based on $\Delta \nu = 8.60 \pm 0.04 \ \mu\rm{Hz}$. As suspected, the spectra are globally interlaced, with the dipole modes of one component close to the radial modes of the other component, and vice versa.}

This value aligns the diagonal structure seen in the \'echelle diagram and satisfies the ($\ell=0,1-\ell=1,0$) near-overlap evident in Figure \ref{fig:echelle}. However, because these peaks are only marginally detected, $\nu_{\rm{max}}$ cannot be measured. The asteroseismic scaling connecting $\Delta\nu$ with the mean density yields $\bar{\rho}/\bar{\rho}_\odot = (4.14 \pm 0.02)\times 10^{-3}$. \revise{This density is larger than we expect; in fact, we expect Star 1 to be less dense than Star 2, the suspected main oscillator.}

We also investigate whether the modes show any frequency modulation as a function of orbital phase by examining portions of the power spectrum spanning less than the orbital period. However, the solar-like oscillations modes are short-lived (about 23 days from an average $0.5 \ \mu\rm{Hz}$ width of $l=0$ modes), so it is difficult to clearly resolve Doppler-shifted modes in a power spectrum of a light curve segment. At $\nu_{\rm{max}} = 106 \ \mu \rm{Hz}$, the maximum frequency shift expected from a $60 \ \rm{km} \ \rm{s}^{-1}$ difference in radial velocity is $0.02 \ \mu \rm{Hz}$. This is less than the intrinsic mode line width, and therefore not observable.
