\subsection{A hint of a second set of oscillations}\label{search}
\label{subsubsec_second_osc}

Given that the giants in KIC 9246715 are nearly twins, with $L_1/L_2 = 0.94$, $R_1/R_2 = 0.95$, $M_1/M_2 = 1.01$, and $T_1/T_2 = 1.01$, we test whether it is possible that we see only one set of oscillation modes because both stars are oscillating with virtually identical frequencies. It is unlikely that two sets of solar-like oscillations lie on top of one another, because the predicted $\nu_{\rm{max}}$ for these not-quite-identical stars are $115.47 \pm 4.46$ and $102.9 \pm 3.46\ \mu$Hz for Star 1 and Star 2 respectively (from an inversion of Equations \ref{density} and \ref{gravity}), and the predicted $\Delta\nu_{\rm{obs}}$ are $8.85 \pm 0.15$ and $8.14 \pm 0.12 \ \mu$Hz for Star 1 and Star 2 respectively. The effective frequency resolution of the power spectrum for four years of \emph{Kepler} data is about $0.008 \ \mu \rm{Hz}$, and, more importantly, the intrinsic observed mode line widths is about $0.5 \ \mu \rm{Hz}$. Given this, the oscillation pattern from a second star (if present) should \emph{not} appear to lie exactly on top of the oscillation pattern we do see.

However, we searched for a second set of oscillations because of the broad, mixed-mode-like appearance of the $l=0$ modes in Figure \ref{fig:echelle}, where mixed modes are not physically possible, and by the faint diagonal structure mostly present on the upper left side of the $l=1$ mode ridge. Even though oscillation modes cannot perfectly overlap, as explained above, modes of degrees $0,1$ of one star can almost overlap modes of degrees $1,0$ of the other. We searched what large frequency spacing can satisfy that condition and match the aligned peaks. For this, we used the universal pattern of red-giant oscillations, which is extensively described in \citet{mos11}, to identify that a $\Delta\nu = 8.60\ \mu$Hz is able to both fit the alignment and overlap the modes of the main oscillator. Because these peaks are marginally detected, no $\nu_{\rm{max}}$ cannot be measured. The asteroseismic scaling connecting $\Delta\nu$ with the mean density leads to $\bar{\rho}$

%To search for the signature of a second set of oscillations, we consider variants of Figure \ref{fig:echelle} with slightly larger $\Delta \nu$ values. This is motivated by the broad, mixed-mode-like appearance of the $l=0$ modes in Figure \ref{fig:echelle}, where mixed modes are not physically possible, and by the faint diagonal structure in both the $l=1$ and $l=0,2$ mode regime. We find that $\Delta \nu \simeq 8.60 \ \mu \rm{Hz}$ reveals a hint of a second set of oscillation modes. Even if these are real and attributable to Star 1, the signal is extremely low-amplitude and cannot be measured robustly.

We also investigate whether the modes show any frequency modulation as a function of orbital phase by examining portions of the power spectrum spanning less than the orbital period. However, the solar-like oscillations modes are short-lived (about 23 days from an average 0.5 $\mu$Hz width of $l=0$ modes), so it is difficult to clearly resolve Doppler-shifted modes in a power spectrum of a light curve segment. At $\nu_{\rm{max}} = 106 \ \mu \rm{Hz}$, the maximum frequency shift expected from a $60 \ \rm{km} \ \rm{s}^{-1}$ difference in radial velocity is $0.02 \ \mu \rm{Hz}$. This is less than the intrinsic mode line width, and therefore not easily observable.

  
  
  
  
  
  
  
  
  
  
  
  
  